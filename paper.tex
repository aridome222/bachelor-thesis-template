%%
% 今時jarticleやjbook使ってる人いる?時代はjsarticleかjsbookだよ
% ついでに言うと、uplatexってのはplatexの上位互換、これを使わないなんて旧世代だよね
%
\documentclass[uplatex, report, a4j, 10pt]{jsbook}


%%
% パッケージ群
%
\usepackage{packages/miyazaki-u-paper}   % 宮崎大学工学部の卒論の基本(片山先生作)を、僕がちょっと書き換えちゃった(テヘッ
\usepackage{enumitem}           % enumerate?古い古い
\usepackage[dvipdfmx]{graphicx} % 当然dvipdfmなんて使ってないよね
\usepackage[dvipdfmx]{color}    % listingsを使うときにはこれも必須、dvipdfmxを変えちゃうとgraphicxのdvipdfmxも変わるよ
\usepackage{listings, packages/jlisting} % コードを埋め込むなら必須
\usepackage{txfonts}            % フォントといえばやっぱりtxfonts、今はnewtxってのもあるらしい
\usepackage{verbatim}           % コメントアウトしてくれる便利なプリアンブルが使える \begin{comment} ... \end{comment}
\usepackage{url}
% \usepackage{easy-todo}
\usepackage[hdivide={21mm, , 21mm}, vdivide={30mm, , 25mm}]{geometry} % スタイルを少し変えたくても\hoffset, \voffsetは使わないでね
\usepackage{multirow}
\usepackage{ascmac}


% \usepackage{latexsym}
% \usepackage{bmpsize}
% \usepackage{comment}


%%% 自分で追加したパッケージ
\usepackage[dvipdfmx]{xcolor}    % listingsを使うときにはこれも必須、dvipdfmxを変えちゃうとgraphicxのdvipdfmxも変わるよ
\usepackage[dvipdfmx, hidelinks]{hyperref} % リンクを付けてくれる。
\usepackage{pxjahyper}          % リンクを付けてくれる(日本語)

%%%


% \def\Underline{\setbox0\hbox\bgroup\let\\\endUnderline}
% \def\endUnderline{\vphantom{y}\egroup\smash{\underline{\box0}}\\}

\newcommand{\ttt}[1]{\texttt{#1}}
\newcommand{\toolName}{MixVRT}  % ツール名を設定

%%
% 論文を書く上で気を付けること
% アンダーバー"_"は"\"でエスケープする

%%
% miyazaki-u-paper.sty用設定値
%
\degree{g} % Graduateのg or Masterのm
\figurenumbering{f} % 図目次を付ける場合はt (真) を持つ真偽値を引数に取る関数
\tablenumbering{f} % 表目次を付ける場合はt (真) を持つ真偽値を引数に取る関数
\title{Webページの画像とHTMLコードにおける\\レイアウトの不具合箇所可視化を目的とした\\視覚的回帰テストツールMixVRTの試作}
\author{有留 直希}
\nendo{5} % 年度
\advisor{片山 徹郎 教授} % 修論では無視する
\major{情報システム工学科}

% \lstset{
% 	label={some_label}
%     backgroundcolor=\color{lightgray},
%     basicstyle=\ttfamily\small,
%     breaklines=true,
%     showstringspaces=false
% }
\lstset{
    language=HTML,
    backgroundcolor=\color{white}, % 背景色を白に設定(またはこの行を完全に削除)
    basicstyle=\ttfamily\small,
    keywordstyle=\color{blue},
    stringstyle=\color{orange},
    commentstyle=\color{green},
    morecomment=[s][\color{green}]{<!}{>},
    showstringspaces=false,
    numbers=left,
    numberstyle=\tiny,
    breaklines=true,
    breakatwhitespace=true,
    tabsize=4,
    frame=single,
    extendedchars=true,
    xleftmargin=2em,
    framexleftmargin=2em
}

\begin{document}
\maketitle

%
% 概要
% 
\preface{概要}
ここに概要を書く。


%
% 本文
% 
\chapter{はじめに}\label{cha:Introduction}
% 本研究では、レイアウトの不具合を可視化する、視覚的回帰テストツール\toolName の試作を行う。
% \toolName は、入力としてWebページのURLをコマンドライン上で受け取り、以下の画像を生成し、Webページ上で表示する。

% また、本研究で用いる「比較画像」、「テスト画像」を、以下に定義する。

Webページの変更前後における画像比較に基づく差分箇所は、それが意図した変更であるかどうかを確認することができない。
特に、テスターは、差分箇所が意図した変更であるかどうかを確認するためには、
HTMLコードを見て得られるコード情報と、画像比較に基づく差分箇所から得られる視覚情報を
頼りにデバッグする必要がある。
このデバッグ作業により、差分箇所が意図した変更であるかどうかを判定できるが、
コード情報と視覚情報間の整合性の確認には、時間がかかる。
\par
そこで本研究では、コード情報と視覚情報間における整合性の確認にかかる時間の削減を目的として、
レイアウトの副作用箇所強調表示(可視化)機能を持つ視覚的回帰テストツール\toolName を試作する。
\par
【下記は2章に書く】\\
画像比較に基づく差分箇所には、意図したレイアウトの変更と意図しないレイアウトの変更がある。
意図したレイアウトの変更はHTMLコードに基づいており、意図しないレイアウトの変更はHTMLコードに基づいていない。
なお、本研究では、意図しないレイアウトの変更箇所を、「レイアウトの副作用箇所」と定義する。

% \par
% 最初に、目視で変更前後のWebページを見比べて、不具合がないかどうかを見つけることの難しさを記述する。
% \begin{itemize}
%     \item Item 01
%     \item Item 02
% \end{itemize}
$\\\\\\\\\\$
画像ベースの視覚的回帰テストの説明と、その問題点について記述する。
\begin{itemize}
    \item Item 01
    \item Item 02
\end{itemize}

本研究で試作するツールで課題を解決できることについて記述する。
\begin{itemize}
    \item Item 01
    \item Item 02
\end{itemize}

\par
本論文の構成を、以下に示す。\par
第2章では、本研究に必要となる前提知識を説明する。\par
第3章では、MixVRTの機能と外観について説明する。\par
第4章では、MixVRTの実装について説明する。\par
第5章では、適用例を用いて、MixVRTが正しく動作することを示す。\par
第6章では、MixVRTについて考察する。\par
第7章では、本研究のまとめと今後の課題について述べる。

\chapter{研究の準備}\label{cha:Preparation}

\section{視覚的回帰テスト}\label{sec:vrt}
視覚的回帰テスト (Visual regression testing)\cite{Visual regression testing}は、
Webページの変更前画像と変更後画像を比較し差分を検出することで、意図しないレイアウトの変更が発生していないことを確認するテスト手法である。
視覚的回帰テストの基本的な手順は以下の通りである。
\begin{enumerate}
    \setlength{\itemsep}{0pt}
          \setlength{\parsep}{0pt}
    \item 変更前画像と変更後画像の作成
    \item 画像比較による差分検出
    \item 結果の評価
\end{enumerate}

\section{レイアウトの不具合}\label{sec:layout effect}
レイアウトの不具合は、Webページの画面要素が適切にレイアウトされていないことである。【TODO: レイアウトの不具合検出に関する関連研究のリンクを見つける】
レイアウトの不具合が発生すると、WebサイトやWebアプリの使いやすさを損ない、ユーザが必要な情報を見つけることが難しくなる。
レイアウトの不具合が発生する主な原因は、画像やテキストの幅や高さを変えたり、他のHTML要素を包含しレイアウトを整理するHTML要素であるコンテナ内の高さを固定値にしたりすることに起因している。
本研究では、以下の主な3つのレイアウトの不具合の発見を支援するツールを試作する。
\begin{itemize}
    \setlength{\itemsep}{0pt}
          \setlength{\parsep}{0pt}
    \item 画面要素の隠れ
    \item 画面要素の重なり
    \item 画面要素のはみ出し
\end{itemize}

\section{OpenCV}\label{sec:OpenCV}
OpenCV (Open Source Computer Vision Library)は、画像や動画に関する処理機能をまとめた、コンピュータビジョン向けのオープンソースのライブラリである\cite{OpenCV}。

\section{Selenium WebDriver}\label{sec:Selenium_WebDriver}
Selenium WebDriver\cite{Selenium WebDriver}は、Seleniumプロジェクト\cite{Selenium}の一部であり、Webブラウザを直接制御するためのAPIを提供する。
Selenium WebDriverのAPIを用いてテストスクリプトを実行することで、指定したWebページにアクセスし、要素のクリックやページ遷移などのユーザ操作を自動で行う。
\section{requestsモジュール}\label{sec:requests}

\section{difflibモジュール}\label{sec:difflib}

\section{Flask}\label{sec:Flask}
\chapter{ \toolName の外観と機能}\label{cha:Function}
本章では、本研究で試作したツール\toolName (Mix Visual Regression Test)の外観と機能について説明する。
\toolName (Mix Visual Regression Test)は、WebページのHTMLコードと画像に基づく視覚的回帰テスト支援ツールである。


\section{外観}\label{sec:area_detection}
\toolName の外観を、図3.1に示す。\toolName は、以下に示す4つのタブメニューボタンと2つのエリアからなる。
なお、以下の数字は、図3.1の数字と対応している。
\begin{itemize}
    \item[①] オリジナル画像表示タブメニューボタン
    \item[②] 画像比較に基づく差分箇所表示タブメニューボタン
    \item[③] HTMLの変更に基づく影響箇所表示タブメニューボタン
    \item[④] レイアウトの副作用箇所表示タブメニューボタン
    \item[⑤] 画像表示エリア
\end{itemize}
\par

\subsection{オリジナル画像表示タブメニューボタン}\label{subsec:original_tab}
オリジナル画像表示タブメニューボタンを押すと、変更前のWebページ画面画像と変更後のWebページ画面画像を表示する。


\subsection{画像比較に基づく差分箇所表示タブメニューボタン}\label{subsec:img_tab}
画像比較に基づく差分箇所表示タブメニューボタンを押すと、画像比較に基づく差分箇所を強調表示した、変更前のWebページ画面画像と変更後のWebページ画面画像を表示する。

\subsection{HTMLの変更に基づく影響箇所表示タブメニューボタン}\label{subsec:html_tab}
HTMLの変更に基づく影響箇所表示タブメニューボタンを押すと、HTMLの変更に基づく影響箇所を強調表示した、変更前のWebページ画面画像と変更後のWebページ画面画像を表示する。


\subsection{レイアウトの副作用箇所表示タブメニューボタン}\label{subsec:subeffect_tab}
レイアウトの副作用箇所画像表示タブメニューボタンを押すと、レイアウトの副作用箇所を強調表示した、変更前のWebページ画面画像と変更後のWebページ画面画像を表示する。


\subsection{画像表示エリア}\label{subsec:img_area}
画像表示エリアは、Webページの変更前と変更後の画像を表示する。


\section{機能}\label{sec:label_detection}
\toolName は、WebページのURLを入力とする。
出力は、


\subsection{画像とHTMLコード取得機能}\label{subsec:a1}


\subsection{差分抽出機能}\label{sec:a2}


\subsection{差分表示機能}\label{sec:a3}

\chapter{MixVRTの実装}\label{cha:Implementation}



\section{画像&HTMLコード取得部}\label{sec:area_detection_part}

\subsection{Seleniumによる画像取得}\label{subsec:rect_detection}

\subsection{requestsによるHTMLコード取得}\label{subsec:underline_detection}


\section{差分抽出部}\label{sec:OCR_part}

\subsection{画像比較による差分抽出}\label{subsec:char_extraction}

\subsection{HTMLの変更による影響箇所抽出}\label{subsec:bbox_coords_obtainment}

\subsection{画像とHTMLコードに基づくレイアウトの副作用抽出}\label{subsec:bbox_obtainment}


\section{差分表示部}\label{sec:label_link_part}
\chapter{適用例}\label{cha:Indication}
本章では、適用例を用いて、今回試作した\toolName が正しく動作することを確認する。
\toolName が変更前のWebページのURLと、変更後のWebページのURLを入力として、
以下に示す8つのPNG形式の画像を生成する。
\begin{itemize}
    \item Webページの変更前画像
    \item Webページの変更後画像
    \item 画像比較に基づく差分箇所を、色付きの枠で囲むことで強調表示した、Webページの変更前画像
    \item 画像比較に基づく差分箇所を、色付きの枠で囲むことで強調表示した、Webページの変更後画像
    \item HTMLコードの変更に基づく変更箇所を、色付きの枠で囲むことで強調表示した、Webページの変更前画像
    \item HTMLコードの変更に基づく変更箇所を、色付きの枠で囲むことで強調表示した、Webページの変更後画像
    \item レイアウトの不具合箇所を、色付きの枠で囲むことで強調表示した、Webページの変更前画像
    \item レイアウトの不具合箇所を、色付きの枠で囲むことで強調表示した、Webページの変更後画像
\end{itemize}
今回の検証を行うために、以下のWebページを用意する。
\begin{itemize}
    \setlength{\itemsep}{0pt}
          \setlength{\parsep}{0pt}
    \item テスト対象とするWebページA\label{item: ex1_bf}
    \item AのWebページに対して、レイアウトの不具合が発生する変更を埋め込んだWebページB\label{item: ex1_af}
\end{itemize}
テスト対象とするWebページAを図\ref{fig:bf_original}に、
図\ref{fig:bf_original}のWebページAにレイアウトの不具合箇所が発生する変更を埋め込んだWebページBを図\ref{fig:af_original}に、
それぞれ示す。
% 以降、上記の2つのWebページを適用例に用いて、レイアウトの不具合箇所を可視化できることを確認する。

\begin{figure}[htbp]
    \centering
    % 画像ファイル名とサイズを指定
    \includegraphics[width=0.5\textwidth]{image/5/original_png/bf_original.png}
    \caption{テスト対象とするWebページA}
    \label{fig:bf_original}
\end{figure}

\begin{figure}[htbp]
    \centering
    % 画像ファイル名とサイズを指定
    \includegraphics[width=0.5\textwidth]{image/5/original_png/af_original.png}
    \caption{図\ref{fig:bf_original}のWebページAにレイアウトの不具合が発生する変更を埋め込んだWebページB}
    \label{fig:af_original}
\end{figure}
上記のWebページを使用することで、
適用例に対して視覚的回帰テストを行った\toolName の各表示タブを確認することができる。

以降の節では、適用例を用いて、\toolName が持つ以下のそれぞれの機能について、確認する。
% 画像比較に基づく差分箇所表示、HTMLコードの変更に基づく変更箇所表示、レイアウトの不具合箇所表示
% のそれぞれの
% 画像比較に基づく差分箇所からHTMLコードに基づかないレイアウトの不具合箇所を
% 可視化できることを確認する。
% \toolName のUIは、以下に示す4つのタブを持つタブメニューと、各タブに対応するタブコンテンツからなる。
% \begin{itemize}
%     \item[①] タブメニュー
%           \begin{itemize}
%               \item オリジナル表示タブ
%               \item 画像比較に基づく差分箇所表示タブ
%               \item HTMLコードの変更に基づく変更箇所表示タブ
%               \item レイアウトの不具合箇所表示タブ
%           \end{itemize}
%     \item[②] タブコンテンツ
% \end{itemize}
% \par
% \toolName を用いて、HTMLコードに基づかないレイアウトの不具合箇所を可視化できるかどうかを確認するために、以下の4つのパターンについて確認する。
% \begin{itemize}
%     \item  画像比較に基づく差分箇所通りにHTMLコードに基づく変更箇所に削除されている
%     \item 画像比較に基づく差分箇所通りにHTMLコードに基づく変更箇所に削除されていない
%     \item 画像比較に基づく差分箇所通りにHTMLコードに基づく変更箇所に追加されている
%     \item 画像比較に基づく差分箇所通りにHTMLコードに基づく変更箇所に追加されていない
% \end{itemize}
% \par

\section{画像比較に基づく差分箇所表示の確認}
適用例を用いて、画像比較に基づく差分箇所表示が、正しく行われていることを確認する。
図\ref{fig:bf_original}と図\ref{fig:af_original}の適用例における画像の比較に基づく差分箇所表示を、図\ref{fig: 5_app2}に示す。
図\ref{fig: 5_app2}より、画像比較に基づく差分箇所表示が、正しく行われていることを確認できる。

\begin{figure}[tp]
    \begin{center}
        \includegraphics[width=1.0\columnwidth]{image/5/5_app2.png}
        \caption{図\ref{fig:bf_original}と図\ref{fig:af_original}における画像比較に基づく差分箇所表示}
        \label{fig: 5_app2}
    \end{center}
\end{figure}



\section{HTMLコードの変更に基づく変更箇所表示の確認}
適用例を用いて、HTMLコードの比較に基づく変更箇所表示が、正しく行われていることを確認する。
図\ref{fig:bf_original}と図\ref{fig:af_original}の適用例における変更箇所表示を、図\ref{fig: 5_app1}に示す。
図\ref{fig: 5_app1}より、HTMLコードの比較に基づく変更箇所表示が、正しく行われていることを確認できる。
\begin{figure}[tp]
    \begin{center}
        \includegraphics[width=1.0\columnwidth]{image/5/5_app1.png}
        \caption{図\ref{fig:bf_original}と図\ref{fig:af_original}の適用例における変更箇所表示}
        \label{fig: 5_app1}
    \end{center}
\end{figure}


\section{レイアウトの不具合箇所表示の確認}
適用例を用いて、レイアウトの不具合箇所表示が、正しく行われていることを確認する。
図\ref{fig:bf_original}と図\ref{fig:af_original}の適用例におけるレイアウトの不具合箇所表示を、図\ref{fig: 5_app3}に示す。
HTMLコードに基づかないレイアウトの不具合箇所を確認するために、以下の4つのパターンについて確認する。
\begin{itemize}
    \item 画像比較に基づく差分箇所通りにHTMLコードに基づく変更箇所に削除されている
    \item 画像比較に基づく差分箇所通りにHTMLコードに基づく変更箇所に削除されていない
    \item 画像比較に基づく差分箇所通りにHTMLコードに基づく変更箇所に追加されている
    \item 画像比較に基づく差分箇所通りにHTMLコードに基づく変更箇所に追加されていない
\end{itemize}
図\ref{fig: 5_app3}より、レイアウトの不具合箇所表示が、正しく行われていることを確認できる。
\begin{figure}[tp]
    \begin{center}
        \includegraphics[width=1.0\columnwidth]{image/5/5_app_3.png}
        \caption{図\ref{fig:bf_original}と図\ref{fig:af_original}の適用例におけるレイアウトの不具合箇所表示}
        \label{fig: 5_app3}
    \end{center}
\end{figure}




\subsection{パターン1: 差分箇所通りに削除されている}\label{sec:result_area_detection}
適用例を用いて、差分箇所通りに削除されていることを確認する。
図\ref{fig: 5_app2}の画面を確認すると、変更前画像上の一番下のリンクが赤枠で囲まれていることが分かる。
この赤枠に着目すると、赤枠は変更後画像上で削除されていることが分かる。
次に、差分箇所通りに削除されているかどうかを確かめるために、
図\ref{fig: 5_app1}のHTMLコードの比較に基づく変更箇所表示タブを確認すると、
赤枠で囲まれているため、HTMLコードに基づいて削除された箇所だと分かる。
最後に、図\ref{fig: 5_app3}のレイアウトの不具合箇所表示タブで確認すると、
レイアウトの不具合箇所として赤枠で囲まれていないことが分かる。
\par
よって、\toolName は、差分箇所通りに削除されていることを確認できる。



\subsection{パターン2: 差分箇所通りに削除されていない}\label{sec:result_area2}
適用例を用いて、差分箇所通りに削除されていないことを確認する。
図\ref{fig: 5_app2}の画面を確認すると、変更前画像上のログインボタンが赤枠で囲まれていることが分かる。
この赤枠に着目すると、変更後画像上で削除されていることが分かる。
次に、差分箇所通りに削除されているかどうかを確かめるために、
図\ref{fig: 5_app1}のHTMLコードの比較に基づく変更箇所表示タブを確認すると、
赤枠で囲まれていないため、HTMLコードに基づいて削除されていない箇所だと分かる。
このことから、緑枠で囲まれたテキストによって、ログインボタンが隠れた状態になっていると推測できる。
最後に、図\ref{fig: 5_app3}のレイアウトの不具合箇所表示タブで確認すると、
レイアウトの不具合箇所として赤枠で囲まれていることが分かる。
\par
よって、\toolName は、差分箇所通りに削除されていないことを確認できた。


\subsection{パターン3: 差分箇所通りに追加されている}\label{sec:result_area3}
適用例を用いて、差分箇所通りに追加されていることを確認する。
図\ref{fig: 5_app2}の画面を確認すると、変更後画像上の左上テキスト「受講科目登録」が緑枠で囲まれていることが分かる。
この緑枠に着目すると、変更前画像上から追加されていることが分かる。
次に、差分箇所通りに追加されているかどうかを確かめるために、
図\ref{fig: 5_app1}のHTMLコードの比較に基づく変更箇所表示タブを確認すると、
緑枠で囲まれているため、HTMLコードに基づいて追加された箇所だと分かる。
最後に、図\ref{fig: 5_app3}のレイアウトの不具合箇所表示タブで確認すると、
レイアウトの不具合箇所として緑枠で囲まれていないことが分かる。
\par
よって、\toolName は、差分箇所通りに追加されていることを確認できた。


\subsection{パターン4: 差分箇所通りに追加されていない}\label{sec:result_area4}
適用例を用いて、差分箇所通りに追加されていないことを確認する。
図\ref{fig: 5_app2}の画面を確認すると、変更後画像上の右上テキスト「登録可能科目一覧」が緑枠で囲まれていることが分かる。
この緑枠に着目すると、緑枠内のテキストと完全一致するテキストが変更前画像上に赤枠で囲まれているため、
変更前画像上から削除され、変更後画像上に新しく追加されていることが分かる。
% つまり、変更後にテキストの配置の変更があったと推測できる。
次に、差分箇所通りに追加されているかどうかを確かめるために、
図\ref{fig: 5_app1}のHTMLコードの比較に基づく変更箇所表示タブを確認すると、
緑枠で囲まれていないため、HTMLコードに基づいて追加されていない箇所だと分かる。
最後に、図\ref{fig: 5_app3}のレイアウトの不具合箇所表示タブで確認すると、
レイアウトの不具合箇所として緑枠で囲まれていることが分かる。
\par
よって、\toolName は、差分箇所通りに削除されていないことを確認できた。
% \section{画面要素の隠れが発生したWebページ}\label{subsec:result_rect_area}
% \subsection{Case1:開発者の意図しないレイアウトの不具合}\label{subsec:result_rect_area}

% \subsection{Case2:開発者が意図して画面要素を消した場合}\label{subsec:result_underline_area}

% \subsection{Case3:開発者が意図せず画面要素を消した場合}\label{subsec:result_underline}


% \section{画面要素の見切れが発生したWebページ}\label{subsec:result_underline_area}
% \subsection{Case1:開発者の意図しないレイアウトの不具合}\label{subsec:result_rect_area}

% \subsection{Case2:開発者が意図して画面要素を消した場合}\label{subsec:result_underline_area}

% \subsection{Case3:開発者が意図せず画面要素を消した場合}\label{subsec:result_underline}

% \section{画面要素の重なりが発生したWebページ}\label{sec:result_area_detection}

% \subsection{Case1:開発者の意図しないレイアウトの不具合}\label{subsec:result_rect_area}

% \subsection{Case2:開発者が意図して画面要素を消した場合}\label{subsec:result_underline_area}

% \subsection{Case3:開発者が意図せず画面要素を消した場合}\label{subsec:result_underline}
\chapter{考察}\label{cha:Discussion}
本研究では、HTMLコードの差分とレイアウトの差分における整合性の確認にかかる時間の削減を目的として、
レイアウトの不具合箇所の可視化機能を持つ視覚的回帰テストツール\toolName を試作した。
本章では、
評価実験を行い、\toolName の有用性を評価する。
次に、\toolName と関連研究を比較する。
最後に、\toolName の問題点とその解決策について述べる。

\section{評価実験}
評価実験では、HTMLコードの差分とレイアウトの差分における整合性の確認にかかる時間に対する評価を行う。
具体的には、\toolName を使用した場合と、一般的な画像ベースの視覚的回帰テストで生成する差分画像を使用する場合とで、
目視によるレイアウトの不具合箇所の発見にかかる時間を、それぞれ計測する。
なお、\toolName を使用する場合をCaseA、一般的な画像ベースの視覚的回帰テストで生成する差分画像を使用する場合をCaseB
とする。

CaseAでは、
\toolName の実行完了後から、被験者がレイアウトの不具合箇所を見つけるまでにかかった時間を発見時間として計測する。
CaseBでは、
変更前後のWebページの差分画像を見てから、被験者がレイアウトの不具合箇所を見つけるまでにかかった時間を発見時間として計測する。


実験の事前準備として、HTMLコードで記述された実験用Webページを2つ作成し、それぞれ変更前と変更後のHTMLコードを用意する。
なお、2つの実験用Webページについて、変更後のWebページには、それぞれ3個のレイアウトの不具合箇所を埋め込む。
1つ目のWebページは、架空の大学サイトのWebページである。これを、サイトAと定義する。
2つ目のWebページは、架空のツール紹介サイトのWebページである。これを、サイトBと定義する。
実験に使用するWebページの例として、変更前後のサイトAを、
図\ref{fig:test1}に示す。
\begin{figure}[tp]
    \centering
    % 画像ファイル名とサイズを指定
    \includegraphics[width=1.0\textwidth]{image/original_test.png}
    \caption{実験に使用する変更前後のサイトA}
    \label{fig:test1}
\end{figure}
実験に参加する被験者は、宮崎大学で情報工学を専攻する4人の学生(以降、A~Dと呼ぶ)である。
実験を行う際には、実験用Webページの変更前画像と変更後画像を見ることができ、
また、実験用Webページの変更前HTMLコードと変更後HTMLコードも見ることができる。
さらに、被験者がレイアウトの不具合箇所を見つけた際に、その不具合箇所の位置を記録しておくための実験用ファイルも用意する。
% 被験者の中には、Webに関する知識がない者も含まれるが、今回は事前に説明する。
% 実験では、【実験によって\toolName が達成したいことを満たせているかを確認するためにこのような作業を提示する】。
% 以降、各作業の内容と結果を説明する。
\par
CaseAでは被験者は、図\ref{fig:test1_subeffect}に示した
\toolName のレイアウトの不具合箇所表示を用いて、
レイアウトの不具合箇所を確認する。
被験者が全てのレイアウトの不具合箇所を実験用ファイルに書き込み終えた場合に、実験を終了する。
\begin{figure}[tp]
    \centering
    % 画像ファイル名とサイズを指定
    \includegraphics[width=1.0\textwidth]{image/5/5_app_3.png}
    \caption{レイアウトの不具合箇所表示}
    \label{fig:test1_subeffect}
\end{figure}

CaseBでは被験者は、図\ref{fig:test1_img}に示した\toolName の画像比較に基づく差分箇所表示を用いて、
レイアウトの不具合箇所を見つける。また、実験用Webページの変更前HTMLコードと変更後HTMLコードも確認することで、
画像比較に基づく差分箇所が、レイアウトの不具合箇所であるかどうかを判定する。
見つけたレイアウトの不具合箇所は実験用ファイルに書き込んでもらい、被験者が全てのレイアウトの不具合箇所を見つけたと判断したら実験を終了する。
\begin{figure}[tp]
    \centering
    % 画像ファイル名とサイズを指定
    \includegraphics[width=1.0\textwidth]{image/5/5_app2.png}
    \caption{画像比較に基づく差分箇所表示}
    \label{fig:test1_img}
\end{figure}
CaseBにおけるレイアウトの不具合箇所を見つける流れについて、以下に示す。
\begin{enumerate}
    \item 画像比較に基づく差分箇所表示で、変更前画像上の全ての赤枠に対して、以下の操作を繰り返す。
          \begin{enumerate}
              \item 赤枠内の画面要素が、実験用Webページの変更後HTMLコードに存在しない、または、CSSクラスの変更を受けている場合、その画面要素は削除、または、変更されたと判定し、次の赤枠の確認に進む。
              \item 上記のそれ以外の場合、その画面要素は削除されていないと判定する。この場合、レイアウトの不具合箇所であるため、実験用ファイルに記録を取る。
          \end{enumerate}
    \item 画像比較に基づく差分箇所表示で、変更後画像上の全ての緑枠に対して、以下の判定を繰り返す。
          \begin{enumerate}
              \item 緑枠内の画面要素が、実験用Webページの変更前HTMLコードに存在しない、または、CSSクラスの変更を受けている場合、その画面要素は追加されたと判定し、次の緑枠の確認に進む。
              \item 上記のそれ以外の場合、その画面要素は追加されていないと判定する。この場合、レイアウトの不具合箇所であるため、実験用ファイルに記録を取る。
          \end{enumerate}
\end{enumerate}


\section{レイアウトの不具合箇所の発見時間に関する評価}\label{subsec:evalue_required_time}
レイアウトの不具合箇所を発見するのにかかった時間についての実験結果を、表\ref{fig: 6_1}に示す。
% 被験者Aは、大学ページを手動で16m 16s、Ariをツールで39s。
% 被験者Bは、大学ページをツールで1m 3s、Ariを手動で13m 40s。
% 被験者Cは、大学ページを手動で11m 54s、Ariをツールで48s。
% 被験者Dは、大学ページをツールで2m 37s、Ariを手動で12m 45s。

表\ref{fig: 6_1}から、\toolName を使用することで、発見時間を平均で12分22秒(90.6\%)削減することができた。
\toolName を使用することで、発見時間を削減できた要因として、CaseBでは差分画像のみしか被験者に提示できないところを、
CaseAではレイアウトの不具合箇所を可視化した画像を
被験者に提示したためであると考える。これにより、\toolName を用いた視覚的回帰テストでは、
レイアウトの変更があった箇所に対して、
HTMLコードを確認してレイアウトの不具合箇所かどうかを確認する手間を削減できたといえる。
よって、\toolName は、レイアウトの不具合箇所の発見にかかる時間の削減に有用である。

\begin{table}[h]
    \centering
    \caption{レイアウトの不具合箇所の発見時間についての実験結果}
    \label{fig: 6_1}
    \begin{tabular}{c||c|c|c|c}
               & \multicolumn{2}{|c|}{\textbf{サイトA}}
               & \multicolumn{2}{|c}{\textbf{サイトB}}                              \\
        \hline \hline
        被験者 & CaseA                                  & CaseB  & CaseA & CaseB    \\
        \hline \hline
        A      & -                                      & 16m16s & 39s   & -        \\
        B      & -                                      & 11m54s & 48s   & -        \\
        C      & 1m3s                                   & -      & -     & 13m40s   \\
        D      & 2m37s                                  & -      & -     & 12m45s   \\
        \hline
        平均   & 1m50s                                  & 14m5s  & 43.5s & 13m12.5s \\
    \end{tabular}
\end{table}


\section{レイアウトの不具合箇所の検出精度に関する評価}\label{subsec:evalue_accuracy}
評価実験において、CaseBで、被験者が
レイアウトの不具合箇所を過剰に検出することや、検出が不足することがあった。
それに対して、CaseAでは、
レイアウトの不具合箇所を過剰に検出することなく、また、不足なく検出することができた。
つまり、CaseAはレイアウトの不具合箇所を過不足なく検出できた。

CaseBにおけるレイアウトの不具合箇所を過剰に検出した数と、検出が不足した数を、表\ref{tb:result_detect}に示す。
なお、表\ref{tb:result_detect}における不具合箇所は、サイトに埋め込んだレイアウトの不具合箇所の総数を意味する。
\begin{table}[ht]
    \centering
    \caption{CaseBにおけるレイアウトの不具合箇所を過剰に検出した数と検出が不足した数}
    \label{tb:result_detect}
    \begin{tabular}{c||c|c|c|c|c}
        サイト                   & 不具合箇所         & 被験者 & 過剰 & 不足 & \\
        \hline \hline
        \multirow{2}{*}{サイトA} & \multirow{2}{*}{3} & A      & 0    & 0    & \\
        \cline{3-6}
                                 &                    & B      & 0    & 2    & \\
        \hline
        \multirow{2}{*}{サイトB} & \multirow{2}{*}{3} & C      & 1    & 1    & \\
        \cline{3-6}
                                 &                    & D      & 2    & 1    & \\
        \hline \hline
    \end{tabular}
\end{table}

CaseBで被験者がレイアウトの不具合箇所を過剰に検出したことについて、考察する。
被験者CとDは、
見た目の変更があった画面要素に対して、HTMLコードに基づく変更がされていたかどうかの
判定を誤ったため、レイアウトの不具合箇所を過剰に検出したと考える。
判定を誤った箇所の変更後の差分画像を、図\ref{fig:app1}に示す。
また、判定を誤った箇所の変更前のHTMLコードによる変更箇所を、図\ref{fig:app2}に示す。
判定を誤った原因として、
画像比較による差分画像では、図\ref{fig:app1}のように、「カスタムレポート」は囲まれておらず、
HTMLコードに基づく変更箇所では、図\ref{fig:app2}のように「カスタムレポート」は囲まれている。
これにより、差分画像しか見れないCaseBの被験者は、「カスタムレポート」を含むHTMLコードの確認まで行う可能性が低くなる。
よって、誤ってレイアウトの不具合箇所であると判定してしまうためだと考える。
図\ref{fig:app1}に示す。
\begin{figure}[ht]
    \centering
    % 画像ファイル名とサイズを指定
    \includegraphics[width=1.0\textwidth]{image/6/app1.png}
    \caption{「カスタムレポート」が囲まれていない差分画像}
    \label{fig:app1}
\end{figure}

\begin{figure}[ht]
    \centering
    % 画像ファイル名とサイズを指定
    \includegraphics[width=1.0\textwidth]{image/6/app2.png}
    \caption{「カスタムレポート」が囲まれているHTMLコードによる変更箇所}
    \label{fig:app2}
\end{figure}

CaseBで被験者がレイアウトの不具合箇所を不足して検出したことについて、考察する。
BからDの被験者らは、レイアウトの不具合箇所として埋め込んだ画面要素のはみ出しに対して、
レイアウトの不具合箇所の検出不足があった。このことから、
画面要素のはみ出しのような見た目に変更が分かりづらい差分があった場合は、画像ベースの視覚的回帰テストでは検出しづらいことが分かった。

以上の評価結果の考察から、\toolName は、一般的な画像ベースの視覚的回帰テストと比較して、
HTMLコードに基づかない変更箇所を可視化できる視覚的回帰テストツールとして有用である。

\section{関連研究}\label{sec:relation_research}
本研究で試作した\toolName と、関連研究を比較する。
Haruto Tannoらは、領域ベースにおける視覚的回帰テストを行った\cite{RegionDetect}。
領域ベースで画像比較することで、本質的な差分と本質的でない差分を含んだ差分画像から、
本質的な差分のみを抽出できることを提案している。
また、塚越らは、GUI要素の階層構造を利用した差分検出方法を提案した\cite{GuiRetrExternal}。
\par
これらの既存研究は、手動で視覚的回帰テストを行う場合よりも、
画像比較による視覚的回帰テストを用いることにより、
レイアウトの不具合がないかどうかを効率的に見つけ出すことを提案している。
しかし、画像比較による差分をすべて抽出するため、抽出した差分には開発者の意図した変更も含まれてしまう。
そのため、HTMLコードの差分とレイアウトの差分における整合性の確認に時間がかかる。

これに対して、\toolName は、画像比較に基づく差分箇所から、HTMLコードの変更に基づく変更箇所を除いた、
レイアウトの不具合箇所のみを表示することができる。
そのため、\toolName は、既存研究と比較して、HTMLコードの差分とレイアウトの差分における整合性の確認にかかる時間の削減に
有用であるといえる。

\section{\toolName の問題点とその解決策}\label{sec:AWSEL_problems}
\toolName の問題点とその解決策について、以下に示す。
\begin{itemize}
    \item 入力対象とするWebページの画面サイズにおける問題:\\
          比較対象とするWebページの画面サイズと、テスト対象とするWebページの画面サイズが同じでないと、
          視覚的回帰テストを行うことができない。常に、画面サイズが同じ大きさだとは限らないため、画面サイズの差に閾値を設定することで、
          その閾値以内なら、画像の調整や比較方法を動的に変更できるようにする必要がある。
    \item 画像比較の制限に関する問題:\\
          一度にできる画像比較は、1回のみである。
          現在の\toolName では、一度に1回しかテストできないため、大規模な開発に有用性があると言えない。
          テスト対象とするWebページだけでなく、そのWebページから派生する別の開発Webページに対しても、視覚的回帰テストを行えるようにしなければ、
          実用的だと言えない。解決策として、Seleniumによるスクレイピング技術を用いたり、開発リポジトリをクローンしておくだけでそのリポジトリにおける開発Webページに対して、
          視覚的回帰テストを自動で実行できる機能を実装する必要がある。
    \item 入力対象とするWebページの背景における問題:\\
          背景が白地でなく、赤色や緑色であったりした場合に、適切に変更箇所に色付きの枠を付けることができない。
          現在の\toolName では、HTMLの変更箇所に赤枠や緑枠をつけて強調表示したり、画像比較で生成した差分画像に赤枠や緑枠をつけて強調表示したりしているが、
          テスト対象とするWebページの背景が白地でなく、赤や緑など強調する色付きの枠と同じであると、適切に強調することができない。
          解決策として、そのようなWebページは視覚的回帰テストの対象としない処理にするか、背景の色や画像内に多く含む色を検出し、その色と区別できる色で変更箇所を強調表示する必要があると考える。
          さらに、他の方法として、色付きの枠をつけて強調表示するのではなく、スライドバーで変更前画像と変更後画像のそのままの表示で切り替えられるようにすることが考えられる。
    \item 入力対象とするWebページのHTML構造における問題:\\
          複雑なHTML構造をしているWebページに対して、視覚的回帰テストがうまくできない。
          現在の\toolName では、HTML構造が一定の形式に沿っていないと適切にHTMLを解析することができず、視覚的回帰テストに用いることのできない、枠付きHTMLコードを生成してしまう。
          この解決策として、HTMLコードの解析をテキストベースで行うのではなく、DOM解析やより高度な解析技術を用いて、HTMLコードの解析が必要となる。
    \item 外部リンクの読み込みにおける問題:\\
          requestsによるHTMLコード取得では、CSSやJavaScriptがリンクで記述されていると、そのリンクを読み込むことができず、適切な枠付きHTMLコードを生成できない。解決策として、
          単一のHTMLコードだけでなく、そのHTMLに用いるCSSやJavascriptのファイルも取得できるようにする必要がある。
    \item レイアウトの不具合箇所の検出方法における問題:\\
          親子関係にある画面要素間に画面要素の重なりが発生しても、その箇所をレイアウトの不具合箇所として強調表示することができない。
          これが起こる原因は、画像比較で差分箇所に付ける枠と、HTML比較で変更箇所に付ける枠が5割以上重なっていない場合に、それらの枠をレイアウトの不具合箇所に付ける枠としているためである。
          解決策として、兄弟関係にある画面要素間に対応するだけでなく、様々なレイアウトの不具合箇所に対して適切な処理を施す必要がある。
\end{itemize}

\chapter{おわりに}\label{cha:Conclusion}
本研究では、HTMLコードの差分とレイアウトの差分における整合性の確認にかかる時間の削減を目的として、
レイアウトの不具合箇所の可視化機能を持つ視覚的回帰テストツール\toolName (Mix Visual Regression Testing)を試作した。
\par
\toolName は、大きく分けて以下の3つを可視化する機能を持つ。
\begin{itemize}
      \item 画像比較に基づく差分箇所:\\
            Webページの変更前画像とWebページの変更後画像を比較して、
            変更前のWebページから削除された画面要素と変更後のWebページに追加された画面要素を可視化する
      \item HTMLコードの変更に基づく変更箇所:\\
            変更前後のWebページのHTMLコードを比較して、
            HTMLコードにおけるbody要素内の変更とstyle要素内の変更のどちらか、
            または両方の変更が適用された画面要素を可視化する
      \item レイアウトの不具合箇所:\\
            Webページの変更前画像と変更後画像で、変更箇所によって、
            HTMLコードを変更していない画面要素にレイアウトの変更があった画面要素を可視化する
\end{itemize}
\par
以下の4つのケースに対して、\toolName を用いて検証することで、画像比較に基づく差分箇所からHTMLコードに基づかないレイアウトの不具合箇所を可視化できることを確認した。
\begin{itemize}
      \item  画面要素が変更前後で開発者の意図した通りに削除された
      \item 画面要素が変更前後で開発者の意図した通りに削除されていない
      \item 画面要素が変更前後で開発者の意図した通りに追加された
      \item 画面要素が変更前後で開発者の意図した通りに追加されていない
\end{itemize}
% 【下記は2章に書く】\\
% 画像比較に基づく差分箇所には、意図したレイアウトの変更と意図しないレイアウトの変更がある。
% 意図したレイアウトの変更はHTMLコードに基づいており、意図しないレイアウトの変更はHTMLコードに基づいていない。
% なお、本研究では、意図しないレイアウトの変更箇所を、「レイアウトの不具合箇所」と定義する。
\par
また、評価実験として、HTMLコードの差分とレイアウトの差分の整合性に関する評価を行うために、\toolName を使用した場合と、
一般的な画像ベースの視覚的回帰テストで、生成する差分画像を使用する場合とで、レイアウトの不具合箇所を発見するのにかかる時間を、それぞれ計測した。
なお、\toolName を使用する場合をCaseA、一般的な画像ベースの視覚的回帰テストで生成する差分画像を使用する場合をCaseB
として、評価実験を行った。
評価実験の結果、レイアウトの不具合箇所を発見するのにかかる時間を、
CaseAは、
CaseBと比べて、レイアウトの不具合箇所の発見時間を、Webサイト1つあたり平均で12分22秒(90.6\%)削減することができた。
さらに、CaseAは、すべてのレイアウトの不具合箇所を過不足なく検出できた。
対して、CaseBでは、レイアウトの不具合箇所を過剰に検出することや、検出が不足することがあった。
この結果から、\toolName は、一般的な画像ベースの視覚的回帰テストと比較して、
画像比較に基づく差分箇所からHTMLコードに基づかないレイアウトの不具合箇所を可視化することで、
差分画像のみを使用するよりも、目視によるレイアウトの不具合箇所の発見を支援できることを確認した。
\par
以上のことから、本研究で試作した \toolName は、
HTMLコードの差分とレイアウトの差分における整合性の確認にかかる時間の削減に有用であるといえる。
\par
以下に、\toolName の今後の課題を示す。
\begin{itemize}
      \item 画像サイズの差に対する閾値の設定:\\
            比較対象とするWebページの画面サイズと、テスト対象とするWebページの画面サイズが同じでないと、
            視覚的回帰テストを行うことができない。常に、画面サイズが同じ大きさだとは限らないため、ある程度の画面サイズの差の閾値を設定して、
            その閾値以内なら、画像の調整や比較方法を動的に変更できるようにする必要がある。
      \item 画像比較の回数に対する拡張:\\
            一度にできる画像比較は、1回のみである。
            現在の\toolName では、一度に1回しかテストできないため、大規模な開発に有用性があると言えない。
            テスト対象とするWebページだけでなく、そのWebページから派生する別の開発Webページに対しても、視覚的回帰テストを行えるようにしなければ、
            実用的だと言えない。解決策として、Seleniumによるスクレイピング技術を用いたり、開発リポジトリをクローンしておくだけでそのリポジトリにおける開発Webページに対して、
            視覚的回帰テストを自動で実行できる機能を実装する必要がある。
      \item 白地以外の背景画像に対応する拡張:\\
            背景が白地でなく、赤色や緑色であったりした場合に、適切に変更箇所に色付きの枠を付けることができない。
            現在の\toolName では、HTMLの変更箇所に赤枠や緑枠をつけて強調表示したり、画像比較で生成した差分画像に赤枠や緑枠をつけて強調表示したりしているが、
            テスト対象とするWebページの背景が白地でなく、赤や緑など強調する色付きの枠と同じであると、適切に強調することができない。
            解決策として、そのようなWebページは視覚的回帰テストの対象としない処理にするか、背景の色や画像内に多く含む色を検出し、その色と区別できる色で変更箇所を強調表示する必要があると考える。
            さらに、他の方法として、色付きの枠をつけて強調表示するのではなく、スライドバーで変更前画像と変更後画像のそのままの表示で切り替えられるようにすることが考えられる。
      \item HTMLコード解析方法の変更による拡張:\\
            複雑なHTML構造をしているWebページに対して、視覚的回帰テストがうまくできない。
            現在の\toolName では、HTML構造が一定の形式に沿っていないと適切にHTMLを解析することができず、視覚的回帰テストに用いることのできない、枠付きHTMLコードを生成してしまう。
            この解決策として、HTMLコードの解析をテキストベースで行うのではなく、DOM解析やより高度な解析技術を用いて、HTMLコードの解析が必要となる。
      \item HTMLコード以外のファイル取得による拡張:\\
            requestsによるHTMLコード取得では、CSSやJavaScriptがリンクで記述されていると、そのリンクを読み込むことができず、適切な枠付きHTMLコードを生成できない。解決策として、
            単一のHTMLコードだけでなく、そのHTMLに用いるCSSやJavascriptのファイルも取得できるようにする必要がある。
      \item レイアウトの不具合箇所の検出アルゴリズムの拡張:\\
            親子関係にある画面要素間に画面要素の重なりが発生しても、その箇所をレイアウトの不具合箇所として強調表示することができない。
            これが起こる原因は、画像比較で差分箇所に付ける枠と、HTML比較で変更箇所に付ける枠が5割以上重なっていない場合に、それらの枠をレイアウトの不具合箇所に付ける枠としているためである。
            解決策として、兄弟関係にある画面要素間に対応するだけでなく、様々なレイアウトの不具合箇所に対して適切な処理を施す必要がある。
\end{itemize}

% \par
% 初期設定を行った状態でテスト対象とするWebページのURLを入力として受け取ることで、
% 以下に示す、\toolName による生成した画像を、Flaskを用いて構築したローカルサーバ上で動作するWebページに、
% 出力し、表示する。
% \begin{itemize}
%     \item Webページの変更前画像と変更後画像
%     \item 画像比較に基づく差分箇所を、色付きの枠で囲むことで強調表示した、Webページの変更前画像と変更後画像
%     \item HTMLコードの変更に基づく影響箇所を、色付きの枠で囲むことで強調表示した、Webページの変更前画像と変更後画像
%     \item レイアウトの不具合箇所を、色付きの枠で囲むことで強調表示した、Webページの変更前画像と変更後画像
% \end{itemize}
% % \par
% 【TODO: ツールの機能の詳細な説明】

% 【TODO: 適用例で確認したこと】

% 【TODO: 実験で分かったこと】

% 【TODO: 今後の課題】





%%
% 謝辞
%
\acknowledgment
本研究を通じて、常に的確なアドバイスや丁寧かつ熱心なご指導をしていただいた、宮崎大学工学部情報システム工学科の片山徹郎教授に心から感謝いたします。

また、本研究を進めるにあたって、多大なご支援を頂きました、codeless株式会社の皆様に感謝申し上げます。

さらに、研究室の先輩方、卒論の相談に付き合っていただいたり、夜通し添削をしていただきありがとうございました。

同期のメンバーとこの1年間頑張ってきたことを一生忘れません。ありがとうございました。

%%
%参考文献
%
\begin{thebibliography}{0}
    \bibitem{Visual regression testing}Visual regression testing: "awesome-regression-testing
    "\\\url{https://github.com/mojoaxel/awesome-regression-testing}\\アクセス日: 2023/01/28.
    \bibitem{OpenCV}OpenCV: "opencv"\\\url{https://github.com/opencv/opencv/blob/master/LICENSE}\\アクセス日: 2023/01/25.
    \bibitem{Python}Python: "Python 3.9.17"\\\url{https://www.python.org/downloads/release/python-3917/}\\アクセス日: 2023/02/01.
    \bibitem{Pillow}Pillow: "Pillow 10.2.0"\\\url{https://pypi.org/project/pillow/}\\アクセス日: 2023/02/01.
    \bibitem{Imageモジュール}Image Module: "Image Module"\\\url{https://pillow.readthedocs.io/en/stable/reference/Image.html}\\アクセス日: 2023/02/01.
    \bibitem{LANCZOS}oneIPL Specification documentation: "Resize with Lanczos InterpolationS"\\\url{https://spec.oneapi.io/oneipl/latest/transform/resize_lanczos.html}\\アクセス日: 2023/02/01.
    
    % \bibitem{膨張処理}マクセルフロンティア株式会社: "2値画像の膨張・収縮①"\\\url{https://www.frontier.maxell.co.jp/blog/posts/21.html}\\アクセス日: 2023/02/01.
    
    \bibitem{Selenium WebDriver}Selenium WebDriver: "WebDriver"\\\url{https://www.selenium.dev/ja/documentation/webdriver/}\\アクセス日: 2023/01/25.
    \bibitem{Selenium}Selenium: "Selenium"\\\url{https://www.selenium.dev/}\\アクセス日: 2023/01/25. 
\end{thebibliography}

% \newpage
% \listoftodos
\end{document}