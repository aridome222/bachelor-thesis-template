\chapter{ \toolName の外観と機能}\label{cha:Function}
本章では、本研究で試作したツール\toolName (Mix Visual Regression Test)の外観と機能について説明する。
\toolName (Mix Visual Regression Test)は、WebページのHTMLコードと画像に基づく視覚的回帰テスト支援ツールである。


\section{外観}\label{sec:area_detection}
\toolName の外観を、図3.1に示す。\toolName は、以下に示す4つのタブメニューボタンと2つのエリアからなる。
なお、以下の数字は、図3.1の数字と対応している。
\begin{itemize}
    \item[①] オリジナル画像表示タブメニューボタン
    \item[②] 画像比較に基づく差分箇所表示タブメニューボタン
    \item[③] HTMLの変更に基づく影響箇所表示タブメニューボタン
    \item[④] レイアウトの副作用箇所表示タブメニューボタン
    \item[⑤] 画像表示エリア
\end{itemize}
\par

\subsection{オリジナル画像表示タブメニューボタン}\label{subsec:original_tab}
オリジナル画像表示タブメニューボタンを押すと、変更前のWebページ画面画像と変更後のWebページ画面画像を表示する。


\subsection{画像比較に基づく差分箇所表示タブメニューボタン}\label{subsec:img_tab}
画像比較に基づく差分箇所表示タブメニューボタンを押すと、画像比較に基づく差分箇所を強調表示した、変更前のWebページ画面画像と変更後のWebページ画面画像を表示する。

\subsection{HTMLの変更に基づく影響箇所表示タブメニューボタン}\label{subsec:html_tab}
HTMLの変更に基づく影響箇所表示タブメニューボタンを押すと、HTMLの変更に基づく影響箇所を強調表示した、変更前のWebページ画面画像と変更後のWebページ画面画像を表示する。


\subsection{レイアウトの副作用箇所表示タブメニューボタン}\label{subsec:subeffect_tab}
レイアウトの副作用箇所画像表示タブメニューボタンを押すと、レイアウトの副作用箇所を強調表示した、変更前のWebページ画面画像と変更後のWebページ画面画像を表示する。


\subsection{画像表示エリア}\label{subsec:img_area}
画像表示エリアは、Webページの変更前と変更後の画像を表示する。


\section{機能}\label{sec:label_detection}
\toolName は、WebページのURLを入力とする。
出力は、


\subsection{画像とHTMLコード取得機能}\label{subsec:a1}


\subsection{差分抽出機能}\label{sec:a2}


\subsection{差分表示機能}\label{sec:a3}
