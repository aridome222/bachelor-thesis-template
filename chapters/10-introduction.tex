\chapter{はじめに}\label{cha:Introduction}

最初に、目視で変更前後のWebページを見比べて、不具合がないかどうかを見つけることの難しさを記述する。
\begin{itemize}
    \item Item 01
    \item Item 02
    \item Item 03
\end{itemize}

画像ベースの視覚的回帰テストの説明と、その問題点について記述する。
\begin{itemize}
    \item Item 01
    \item Item 02
    \item Item 03
\end{itemize}

本研究で試作するツールで課題を解決できることについて記述する。
\begin{itemize}
    \item Item 01
    \item Item 02
    \item Item 03
\end{itemize}

\par
本論文の構成を、以下に示す。\par
第2章では、本研究に必要となる前提知識を説明する。\par
第3章では、MixVRTの機能と外観について説明する。\par
第4章では、MixVRTの実装について説明する。\par
第5章では、適用例を用いて、MixVRTが正しく動作することを示す。\par
第6章では、MixVRTについて考察する。\par
第7章では、本研究のまとめと今後の課題について述べる。



% \begin{figure}[t]
%     \begin{center}
%         \includegraphics[width=7cm]{image/sample.png}
%         \caption{sample image}
%         \label{fig:sample}
%     \end{center}
% \end{figure}