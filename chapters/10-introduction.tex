\chapter{はじめに}\label{cha:Introduction}
% 本研究では、レイアウトの不具合を可視化する、視覚的回帰テストツール\toolName の試作を行う。
% \toolName は、入力としてWebページのURLをコマンドライン上で受け取り、以下の画像を生成し、Webページ上で表示する。

% また、本研究で用いる「比較画像」、「テスト画像」を、以下に定義する。

Webページの変更前後における画像比較に基づく差分箇所は、それが意図した変更であるかどうかを確認することができない。
特に、テスターは、差分箇所が意図した変更であるかどうかを確認するためには、
HTMLコードを見て得られるコード情報と、画像比較に基づく差分箇所から得られる視覚情報を
頼りにデバッグする必要がある。
このデバッグ作業により、差分箇所が意図した変更であるかどうかを判定できるが、
コード情報と視覚情報間の整合性の確認には、時間がかかる。
\par
そこで本研究では、コード情報と視覚情報間における整合性の確認にかかる時間の削減を目的として、
レイアウトの副作用箇所強調表示(可視化)機能を持つ視覚的回帰テストツール\toolName を試作する。
\par
【下記は2章に書く】\\
画像比較に基づく差分箇所には、意図したレイアウトの変更と意図しないレイアウトの変更がある。
意図したレイアウトの変更はHTMLコードに基づいており、意図しないレイアウトの変更はHTMLコードに基づいていない。
なお、本研究では、意図しないレイアウトの変更箇所を、「レイアウトの副作用箇所」と定義する。

% \par
% 最初に、目視で変更前後のWebページを見比べて、不具合がないかどうかを見つけることの難しさを記述する。
% \begin{itemize}
%     \item Item 01
%     \item Item 02
% \end{itemize}
$\\\\\\\\\\$
画像ベースの視覚的回帰テストの説明と、その問題点について記述する。
\begin{itemize}
    \item Item 01
    \item Item 02
\end{itemize}

本研究で試作するツールで課題を解決できることについて記述する。
\begin{itemize}
    \item Item 01
    \item Item 02
\end{itemize}

\par
本論文の構成を、以下に示す。\par
第2章では、本研究に必要となる前提知識を説明する。\par
第3章では、MixVRTの機能と外観について説明する。\par
第4章では、MixVRTの実装について説明する。\par
第5章では、適用例を用いて、MixVRTが正しく動作することを示す。\par
第6章では、MixVRTについて考察する。\par
第7章では、本研究のまとめと今後の課題について述べる。
