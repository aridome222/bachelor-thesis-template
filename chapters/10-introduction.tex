\chapter{はじめに}\label{cha:Introduction}
%% 修正案 用語を変えない版
% インターネットの利用率は年々増加している\cite{Soumusyou}。それに伴い、WebサイトやWebアプリケーション開発の需要も高まっている。
% 近年のWebアプリケーション開発では、JavaScriptなどのプログラミング言語を用いて、フロントエンドの開発を行うことが一般的であり、
% そのソースコードをHTMLコードに変換することで、Webブラウザ上にページを描画する。
% この際、描画されるページのHTMLコードに変更がないにも関わらず、画面上のHTML要素が変更されてしまう問題がある。
% HTMLコードの変更よる意図した差分だけでなく、HTMLコードの変更によらない意図しない差分が紛れ込む可能性がある。
% この問題に対処するためには、Webアプリケーションの変更前後で、HTMLコードの差分とレイアウトの差分における整合性を確認する必要がある。
% \par
% HTMLコードの差分とレイアウトの差分における整合性を確認する手法の1つに、
% 視覚的回帰テスト(Visual Regression Test)がある。
% 視覚的回帰テストは、開発者がWebページの要素を追加・削除・変更した際に、
% 意図しない箇所でレイアウトの変更が発生していないことを確認するテスト手法である。
% 視覚的回帰テストを行うことで、
% 変更後のWebページを確認するだけでは発見することが困難なレイアウトにおける不具合を、
% 検出することが容易となる。

% しかし、既存の視覚的回帰テストを対象とした研究では、
% 画面上のすべての差分を表示する。この差分には、開発者の意図した変更も含まれるため、
% 表示された差分のうち、どの箇所が整合性が取れていない箇所かを確認する際には、
% HTMLコードを見て判断する必要がある。
% そのため、
% HTMLコードの差分とレイアウトの差分における整合性の確認に手間と時間がかかる。
% \par
% そこで本研究では、
% HTMLコードの差分とレイアウトの差分における整合性の確認にかかる時間の削減を目的として、
% レイアウトの不具合箇所の可視化機能を持つ視覚的回帰テストツール\toolName(Mix Visual Regression Testing)を試作する。
% 画面内 GUI 要素に微小なズレが存在した場合,差分画像に余計な差分が入
% り込んでしまい,テスターが本来検出すべき,アプリケーション不具合による差分を確認
% することが難しくなってしまう.
% 本質的な差分を検出する
%%% 修正案 用語を変えない版
インターネットの利用率は年々増加している\cite{Soumusyou}。
それに伴い、WebサイトやWebアプリケーションといったWebページの需要が高まっている。
\par
近年、Webページは、ユーザ体験(UX:User Experience)やSEO対策の観点から、高い更新頻度が求められている\cite{SEO}。
しかし、Webページの更新のための変更には、HTMLコードの変更による意図したレイアウトの差分だけでなく、HTMLコードの変更によらない意図しないレイアウトの差分が紛れ込む可能性がある。
本研究では、開発者が意図しないレイアウトの差分を、レイアウトの不具合と呼ぶ。
\par
この問題を解決するために、視覚的回帰テスト(Visual Regression Test)\cite{VisualRegressionTesting}がある。
Webページを対象とした視覚的回帰テストでは、変更前後のWebページを並べて、レイアウトの差分を強調表示することで、レイアウトの差分を発見しやすくする。
既存の視覚的回帰テストを対象とした研究として、領域ベースにおける視覚的回帰テストを行う研究\cite{RegionDetect}やGUI要素の階層構造を利用した差分検出方法を提案した研究\cite{GuiRetrExternal}が存在する。
しかし、これらの研究は、変更前後の画像のみを比較し、開発者の意図したレイアウトの差分を含む、すべてのレイアウトの差分を強調表示するものが多い。
レイアウトの不具合を発見するためには、表示されたすべてのレイアウトの差分とHTMLコードを照らし合わせながら、その差分が開発者の意図に沿っているかどうかを判断する必要があり、手間と時間がかかる。
\par
そこで本研究では、Webページのレイアウトの不具合箇所の発見にかかる時間の削減を目的として、
Webページのレイアウト不具合箇所を強調表示する視覚的回帰テストツール\toolName(Mix Visual Regression Testing)を試作する。
\toolName は、以下の3つを強調表示する機能を持つ。
\begin{itemize}
      \item 差分箇所:\\
            変更前後のWebページの画像を比較して、変更前のWebページから削除された範囲と、
            変更後のWebページに追加された範囲。
      \item 変更箇所:\\
            変更前後のWebページのHTMLコードを比較して、HTMLコードにおけるbody要素内の変更とstyle要素内の変更のどちらか、
            または両方が適用された画面要素の範囲。
            本研究では、変更箇所を意図的なレイアウトの差分の範囲とみなす。
      \item レイアウトの不具合箇所:\\
            差分箇所から、意図的なレイアウトの差分の範囲であるHTMLコードの変更箇所を除いた、
            レイアウトの不具合の範囲。
\end{itemize}

% インターネットの利用率は年々増加している\cite{Soumusyou}。
% それに伴い、WebサイトやWebアプリケーションといったWebコンテンツの需要も高まっている。
% 近年、Webコンテンツは、ユーザ体験(UX:User Experience)やSEO対策の観点から、高い更新頻度が求められている\cite{TODO}。
% しかし、Webコンテンツの更新には、HTMLコードの変更よる意図した差分だけでなく、HTMLコードの変更によらない意図しない差分が紛れ込む可能性がある。
% この問題を解決するために、視覚的回帰テスト(Visual Regression Test)\cite{VisualRegressionTesting}がある。
% 視覚的回帰テストでは、変更前後のWebページを並べて、Webコンテンツの差分を強調することで、意図しない差分を発見しやすくする。
% 既存の視覚的回帰テストを対象とした研究として、A\cite{RegionDetect}とB\cite{GuiRetrExternal}がある。
% しかし、これらの研究は、変更前後の画像のみを比較し、開発者の意図した差分も含むすべての差分を強調するものが多い。
% 意図しない差分を特定するには、表示されたすべての差分とHTMLコードを照らし合わせながら、その差分が意図に沿っているかどうかを判断する必要があり、手間と時間がかかる。
% そこで本研究では、意図しない差分の特定にかかる時間の削減を目的として、
% 意図しない差分の可視化機能を持つ視覚的回帰テストツール\toolName(Mix Visual Regression Testing)を試作する。
% \toolName は、以下の3つを可視化する機能を持つ。
% \begin{itemize}
%       \item 差分箇所:\\
%             Webページの変更前画像とWebページの変更後画像を比較して、
%             変更前のWebページから削除された範囲と変更後のWebページに追加された範囲を可視化する。
%       \item 変更箇所:\\
%             変更前後のWebページのHTMLコードを比較して、
%             HTMLコードにおけるbody要素内の変更とstyle要素内の変更のどちらか、
%             または両方の変更が適用された画面要素の範囲を可視化する。
%             本研究では、変更箇所を意図的に変更した範囲と見なす。
%       \item レイアウトの不具合箇所:\\
%             差分箇所から、意図的に変更した範囲であるHTMLコードの変更箇所を除いた、
%             意図しない差分の範囲を可視化する。
% \end{itemize}
% 本論文では、この問題を不整合が発生する場合がある。

% ソフトウェアの大規模化や複雑化に伴い、ソフトウェアの信頼性を確保するために、
% 最初のリリース時だけでなく、
% バージョンアップするたびに、テストを行う必要がある\cite{RegionDetect}。
% この際、バージョンアップ前に動いていた機能や性能が損なわれていないかを確認するテスト手法として、回帰テストが存在する。
% 特に、WebアプリケーションのようなGUIアプリケーションの回帰テストでは、アプリケーション画面が正しく表示されていることを、
% 認する回帰テストである視覚的回帰テスト(Visual Regression Test)がある。
% バージョンアップによるレイアウトの変更に伴う、レイアウトの崩れ
% \par
% Webアプリ開発では、HTML要素を変更していないのに、画面上でそのHTML要素が変更されてしまうというレイアウトの不具合がある。
% 上記のレイアウトの不具合を見つけるには、HTMLコードの差分とレイアウトの差分との間における整合性の確認が必要である。






% Webページの変更前後における画像比較に基づく差分箇所は、開発者の意図した変更であるかどうかを確認することができない。
% 差分箇所が意図した変更であるかどうかを開発者が確認するためには、
% 画像比較に基づく差分箇所から得られる視覚情報と、HTMLコードを見て得られるコード情報を
% 組み合わせることが効果的である。
% このデバッグ作業により、差分箇所が意図した変更であるかどうかを判定できるが、
% コード情報と視覚情報間の整合性の確認には、時間がかかる。
% \par
% そこで本研究では、コード情報と視覚情報間における整合性の確認にかかる時間の削減を目的として、
% レイアウトの副作用箇所強調表示(可視化)機能を持つ視覚的回帰テストツール\toolName を試作する。
% \par
% 【下記は2章に書く】\\
% 画像比較に基づく差分箇所には、意図したレイアウトの変更と意図しないレイアウトの変更がある。
% 意図したレイアウトの変更はHTMLコードに基づいており、意図しないレイアウトの変更はHTMLコードに基づいていない。
% なお、本研究では、意図しないレイアウトの変更箇所を、「レイアウトの不具合箇所」と定義する。

% \par
% 最初に、目視で変更前後のWebページを見比べて、不具合がないかどうかを見つけることの難しさを記述する。
% \begin{itemize}
%     \item Item 01
%     \item Item 02
% \end{itemize}
% $\\\\\\\\\\$
% 画像ベースの視覚的回帰テストの説明と、その問題点について記述する。
% \begin{itemize}
%     \item Item 01
%     \item Item 02
% \end{itemize}

% 本研究で試作するツールで課題を解決できることについて記述する。
% \begin{itemize}
%     \item Item 01
%     \item Item 02
% \end{itemize}

\par
本論文の構成を、以下に示す。\par
第2章では、本研究に必要となる前提知識を説明する。\par
第3章では、MixVRTの機能と外観について説明する。\par
第4章では、MixVRTの実装について説明する。\par
第5章では、適用例を用いて、MixVRTが正しく動作することを示す。\par
第6章では、MixVRTについて考察する。\par
第7章では、本研究のまとめと今後の課題について述べる。
