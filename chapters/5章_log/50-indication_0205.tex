\chapter{適用例}\label{cha:Indication}
本章では、今回試作した\toolName を用いて、画像比較に基づく差分箇所からHTMLコードに基づかないレイアウトの副作用箇所を可視化できることを確認する。
\toolName のUIは、以下に示す4つのタブを持つタブメニューと、各タブに対応するタブコンテンツからなる。
\begin{itemize}
    \item[①] タブメニュー
          \begin{itemize}
              \item オリジナル表示タブ
              \item 画像比較に基づく差分箇所表示タブ
              \item HTMLコードの変更に基づく影響箇所表示タブ
              \item レイアウトの副作用箇所表示タブ
          \end{itemize}
    \item[②] タブコンテンツ
\end{itemize}
\par
\toolName がレイアウトの副作用を可視化できることを確認するために、以下のWebページを用意する。
\begin{itemize}
    \setlength{\itemsep}{0pt}
          \setlength{\parsep}{0pt}
    \item テスト対象とするWebページ\label{item: ex1_bf}
    \item 1.のWebページに対して、画面要素の追加による、Webページ\label{item: ex1_af}
\end{itemize}
テスト対象とするWebページを図\ref{fig:bf_original}に、
図\ref{fig:bf_original}のWebページにレイアウトの副作用が発生する変更を埋め込んだWebページを図\ref{fig:af_original}に、
それぞれ示す。
以降、上記の2つのWebページを適用例に用いて、画面要素の追加によるレイアウトの副作用を可視化できることを確認する。

\begin{figure}[htbp]
    \centering
    % 画像ファイル名とサイズを指定
    \includegraphics[width=0.5\textwidth]{image/5/original_png/bf_original.png}
    \caption{テスト対象とするWebページ}
    \label{fig:bf_original}
\end{figure}

\begin{figure}[htbp]
    \centering
    % 画像ファイル名とサイズを指定
    \includegraphics[width=0.5\textwidth]{image/5/original_png/af_original.png}
    \caption{図\ref{fig:bf_original}のWebページにレイアウトの副作用が発生する変更を埋め込んだWebページ}
    \label{fig:af_original}
\end{figure}

\section{画面要素の追加によるレイアウトの副作用}\label{sec:result_area_detection}
適用例を用いて、\toolName が、画面要素の追加によるレイアウトの副作用を可視化できることを確認する。
\toolName を用いて、以下の3つの操作を行う。
\begin{enumerate}[label=操作\arabic*., leftmargin=1.8cm]
    \item 図\ref{fig:bf_original}のWebページのURLを入力として、コマンドライン上で\toolName の実行コマンド(\ref{subsec:MixVRT_preparation}を参照)を実行する。
    \item 図\ref{fig:af_original}のWebページのURLを入力として、コマンドライン上で\toolName の実行コマンドを実行する。
    \item Webブラウザ上で「http://localhost:5000/MixVRT\_diff」にアクセスする。
\end{enumerate}
\par
上記の操作を行うことで、
適用例に対して視覚的回帰テストを行った\toolName の各表示タブを確認することができる。
図\ref{fig: ex1_html}に、適用例に対するオリジナル表示タブを示す。
オリジナル表示タブは画像を表示するだけで
\begin{figure}[tp]
    \begin{center}
        \includegraphics[width=1.0\columnwidth]{image/5/original_png/original.png}
        \caption{オリジナル表示タブで変更前画像として表示する画像}
        \label{fig: bf_original}
    \end{center}
\end{figure}

\par
レイアウトの副作用
まず、レイアウトの不具合が発生していないかを確認するために、レイアウトの不具合箇所表示タブを確認する。
レイアウトの不具合箇所表示タブにおける、画面要素の隠れが発生していないWebページの画像と画面要素の隠れが発生したWebページの画像を、
図\ref{fig: ex1_subeffect}に示す。
\begin{figure}[tp]
    \begin{center}
        \includegraphics[width=1.0\columnwidth]{image/5/ex1_subeffect.png}
        \caption{オリジナル表示タブで変更後画像として表示する画像}
        \label{fig: ex1_subeffect}
    \end{center}
\end{figure}
図\ref{fig: ex1_subeffect}を見ると、画面要素の隠れが発生していないWebページの画像上に赤枠で囲まれた画面要素が存在するが、
その画面要素と一致する緑枠で囲まれた画面要素は、画面要素の隠れが発生したWebページの画像上には存在しない。
このことから、画面要素の隠れが発生していると推測できる。
次に、赤枠で囲まれた画面要素がHTMLコードの変更による影響を受けているのかを確認する。
HTMLコードの変更による影響箇所表示タブにおける、画面要素の隠れが発生していないWebページの画像と画面要素の隠れが発生したWebページの画像を、
図\ref{fig: ex1_html}に示す。
\begin{figure}[tp]
    \begin{center}
        \includegraphics[width=1.0\columnwidth]{image/5/ex1_html.png}
        \caption{HTMLコードに基づく影響箇所表示タブにおける、画面要素の隠れが発生していないWebページの画像と画面要素の隠れが発生したWebページの画像}
        \label{fig: ex1_html}
    \end{center}
\end{figure}
図\ref{fig: ex1_html}を見ると、レイアウトの不具合箇所表示タブでは赤枠で囲まれた画面要素が、
図\ref{fig: ex1_html}の画面要素の隠れが発生していないWebページの画像上には存在しない。
つまり、赤枠で囲まれた画面要素は、HTMLコードによる影響を受けて
以上より、\toolName は、画面要素の隠れを可視化できることを確認できた。
% オリジナル表示タブにおける、画面要素の隠れが発生していないWebページの画像と画面要素の隠れが発生したWebページの画像を、
% 図\ref{fig: ex1_original}に示す。
% \begin{figure}[tp]
%     \begin{center}
%         \includegraphics[width=1.0\columnwidth]{image/5/ex1_original.png}
%         \caption{オリジナル表示タブにおける、画面要素の隠れが発生していないWebページの画像と画面要素の隠れが発生したWebページの画像}
%         \label{fig: ex1_original}
%     \end{center}
% \end{figure}
% 次に、画像比較に基づく差分箇所表示タブにおける、画面要素の隠れが発生していないWebページの画像と画面要素の隠れが発生したWebページの画像を、
% 図\ref{fig: ex1_img}に示す。
% \begin{figure}[tp]
%     \begin{center}
%         \includegraphics[width=1.0\columnwidth]{image/5/ex1_img.png}
%         \caption{画像比較に基づく差分箇所表示タブにおける、画面要素の隠れが発生していないWebページの画像と画面要素の隠れが発生したWebページの画像}
%         \label{fig: ex1_img}
%     \end{center}
% % \end{figure}
% その次に、HTMLコードに基づく影響箇所表示タブにおける、画面要素の隠れが発生していないWebページの画像と画面要素の隠れが発生したWebページの画像を、
% 図\ref{fig: ex1_html}に示す。
% \begin{figure}[tp]
%     \begin{center}
%         \includegraphics[width=1.0\columnwidth]{image/5/ex1_html.png}
%         \caption{HTMLコードに基づく影響箇所表示タブにおける、画面要素の隠れが発生していないWebページの画像と画面要素の隠れが発生したWebページの画像}
%         \label{fig: ex1_html}
%     \end{center}
% \end{figure}

% 図\ref{fig: ex1_original}より、画面要素の隠れが発生していないWebページの画像と画面要素の隠れが発生したWebページの画像を確認できる。
% 図\ref{fig: ex1_img}により、レイアウトの変更があった画面要素を
図\ref{fig: ex1_subeffect}を見ると、変更前画像上に赤枠で囲んだ画面要素が存在するが、その画面要素と一致する、緑枠で囲まれた画面要素は存在しない。
このことから、画面要素の隠れが発生していると判断できる。
\par
以上のことから、\toolName は、画面要素の隠れを可視化できることを確認できた。
% \toolName に対して、画面要素の隠れを埋め込む前のWebページのURLを入力として実行した時に取得した変更前画像と、
% 画面要素の隠れを埋め込んだ後のWebページのURLを入力として実行した時に取得した変更後画像を
% 画面要素の隠れを埋め込む前のWebページの画像と埋め込んだ後のWebページの画像を、図\ref{fig: ex1_original}に示す。

\section{画面要素の見切れの可視化}\label{sec:result_area_detec}
適用例を用いて、\toolName が画面要素の見切れを、可視化できることを確認する。
\toolName は、画面要素の見切れを埋め込んだ変更前後のWebページのURLをそれぞれ入力として受け取り、
図\ref{fig: ex4_original}に対して画面要素の見切れを可視化した変更前画像と変更後画像を生成する。
% 図\ref{fig: ex1_original}を見ると、変更前画像と変更後画像を比較して、コンテナ内に
画面要素の見切れを可視化した変更前画像と変更後画像を、図\ref{fig: ex4_subeffect}に示す。
\begin{figure}[tp]
    \begin{center}
        \includegraphics[width=1.0\columnwidth]{image/5/ex4_original.png}
        \caption{画面要素の見切れを埋め込んだWebページの変更前画像と変更後画像}
        \label{fig: ex4_original}
    \end{center}
\end{figure}
\begin{figure}[tp]
    \begin{center}
        \includegraphics[width=1.0\columnwidth]{image/5/ex4_subeffect.png}
        \caption{図\ref{fig: ex4_original}に対して画面要素の見切れを可視化した変更前画像と変更後画像}
        \label{fig: ex4_subeffect}
    \end{center}
\end{figure}
図\ref{fig: ex4_subeffect}を見ると、変更前画像上に赤枠で囲んだ画面要素が存在し、その画面要素の一部と一致する、緑枠で囲まれた画面要素が存在する。
このことから、画面要素の見切れが発生していると判断できる。
\par
以上のことから、\toolName は、画面要素の見切れを可視化できることを確認できた。

\section{画面要素の重なりの可視化}\label{sec:result_area_detecti}
適用例を用いて、\toolName が画面要素の重なりを、可視化できることを確認する。
\toolName は、画面要素の重なりを埋め込んだ変更前後のWebページのURLをそれぞれ入力として受け取り、
図\ref{fig: ex3_original}に対して画面要素の重なりを可視化した変更前画像と変更後画像を生成する。
% 図\ref{fig: ex1_original}を見ると、変更前画像と変更後画像を比較して、コンテナ内に
画面要素の重なりを可視化した変更前画像と変更後画像を、図\ref{fig: ex3_subeffect}に示す。
\begin{figure}[tp]
    \begin{center}
        \includegraphics[width=1.0\columnwidth]{image/5/ex3_original.png}
        \caption{画面要素の重なりを埋め込んだWebページの変更前画像と変更後画像}
        \label{fig: ex3_original}
    \end{center}
\end{figure}
\begin{figure}[tp]
    \begin{center}
        \includegraphics[width=1.0\columnwidth]{image/5/ex3_subeffect.png}
        \caption{図\ref{fig: ex3_original}に対して画面要素の重なりを可視化した変更前画像と変更後画像}
        \label{fig: ex3_subeffect}
    \end{center}
\end{figure}
図\ref{fig: ex3_subeffect}を見ると、変更後画像上にテキストとフッターの2つの画面要素が緑枠内で重なっていることが確認できる。
また、変更前画像上には、先ほど述べたテキストとフッターが赤枠で囲まれていることから、それらの画面要素が重なってできた画面要素の重なりであると確認できる。
このことから、画面要素の重なりが発生していると判断できる。
\par
以上のことから、\toolName は、画面要素の重なりを可視化することを確認できた。


% \section{画面要素の隠れが発生したWebページ}\label{subsec:result_rect_area}
% \subsection{Case1:開発者の意図しないレイアウトの不具合}\label{subsec:result_rect_area}

% \subsection{Case2:開発者が意図して画面要素を消した場合}\label{subsec:result_underline_area}

% \subsection{Case3:開発者が意図せず画面要素を消した場合}\label{subsec:result_underline}


% \section{画面要素の見切れが発生したWebページ}\label{subsec:result_underline_area}
% \subsection{Case1:開発者の意図しないレイアウトの不具合}\label{subsec:result_rect_area}

% \subsection{Case2:開発者が意図して画面要素を消した場合}\label{subsec:result_underline_area}

% \subsection{Case3:開発者が意図せず画面要素を消した場合}\label{subsec:result_underline}

% \section{画面要素の重なりが発生したWebページ}\label{sec:result_area_detection}

% \subsection{Case1:開発者の意図しないレイアウトの不具合}\label{subsec:result_rect_area}

% \subsection{Case2:開発者が意図して画面要素を消した場合}\label{subsec:result_underline_area}

% \subsection{Case3:開発者が意図せず画面要素を消した場合}\label{subsec:result_underline}