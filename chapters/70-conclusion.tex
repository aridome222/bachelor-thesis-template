\chapter{おわりに}\label{cha:Conclusion}
本研究では、HTMLコードの差分とレイアウトの差分における整合性の確認にかかる時間の削減を目的として、
レイアウトの不具合箇所の可視化機能を持つ視覚的回帰テストツール\toolName (Mix Visual Regression Testing)を試作した。
\par
\toolName は、大きく分けて以下の3つを可視化する機能を持つ。
\begin{itemize}
      \item 画像比較に基づく差分箇所:\\
            Webページの変更前画像とWebページの変更後画像を比較して、
            変更前のWebページから削除された画面要素と変更後のWebページに追加された画面要素を可視化する
      \item HTMLコードの変更に基づく変更箇所:\\
            変更前後のWebページのHTMLコードを比較して、
            HTMLコードにおけるbody要素内の変更とstyle要素内の変更のどちらか、
            または両方の変更が適用された画面要素を可視化する
      \item レイアウトの不具合箇所:\\
            Webページの変更前画像と変更後画像で、変更箇所によって、
            HTMLコードを変更していない画面要素にレイアウトの変更があった画面要素を可視化する
\end{itemize}
\par
以下の4つのケースに対して、\toolName を用いて検証することで、画像比較に基づく差分箇所からHTMLコードに基づかないレイアウトの不具合箇所を可視化できることを確認した。
\begin{itemize}
      \item  画面要素が変更前後で開発者の意図した通りに削除された
      \item 画面要素が変更前後で開発者の意図した通りに削除されていない
      \item 画面要素が変更前後で開発者の意図した通りに追加された
      \item 画面要素が変更前後で開発者の意図した通りに追加されていない
\end{itemize}
% 【下記は2章に書く】\\
% 画像比較に基づく差分箇所には、意図したレイアウトの変更と意図しないレイアウトの変更がある。
% 意図したレイアウトの変更はHTMLコードに基づいており、意図しないレイアウトの変更はHTMLコードに基づいていない。
% なお、本研究では、意図しないレイアウトの変更箇所を、「レイアウトの不具合箇所」と定義する。
\par
また、評価実験として、HTMLコードの差分とレイアウトの差分の整合性に関する評価を行うために、\toolName を使用した場合と、
一般的な画像ベースの視覚的回帰テストで、生成する差分画像を使用する場合とで、レイアウトの不具合箇所を発見するのにかかる時間を、それぞれ計測した。
なお、\toolName を使用する場合をCaseA、一般的な画像ベースの視覚的回帰テストで生成する差分画像を使用する場合をCaseB
として、評価実験を行った。
評価実験の結果、レイアウトの不具合箇所を発見するのにかかる時間を、
CaseAは、
CaseBと比べて、レイアウトの不具合箇所の発見時間を、Webサイト1つあたり平均で12分22秒(90.6\%)削減することができた。
さらに、CaseAは、すべてのレイアウトの不具合箇所を過不足なく検出できた。
対して、CaseBでは、レイアウトの不具合箇所を過剰に検出することや、検出が不足することがあった。
この結果から、\toolName は、一般的な画像ベースの視覚的回帰テストと比較して、
画像比較に基づく差分箇所からHTMLコードに基づかないレイアウトの不具合箇所を可視化することで、
差分画像のみを使用するよりも、目視によるレイアウトの不具合箇所の発見を支援できることを確認した。
\par
以上のことから、本研究で試作した \toolName は、
HTMLコードの差分とレイアウトの差分における整合性の確認にかかる時間の削減に有用であるといえる。
\par
以下に、\toolName の今後の課題を示す。
\begin{itemize}
      \item 画像サイズの差に対する閾値の設定:\\
            比較対象とするWebページの画面サイズと、テスト対象とするWebページの画面サイズが同じでないと、
            視覚的回帰テストを行うことができない。常に、画面サイズが同じ大きさだとは限らないため、ある程度の画面サイズの差の閾値を設定して、
            その閾値以内なら、画像の調整や比較方法を動的に変更できるようにする必要がある。
      \item 画像比較の回数に対する拡張:\\
            一度にできる画像比較は、1回のみである。
            現在の\toolName では、一度に1回しかテストできないため、大規模な開発に有用性があると言えない。
            テスト対象とするWebページだけでなく、そのWebページから派生する別の開発Webページに対しても、視覚的回帰テストを行えるようにしなければ、
            実用的だと言えない。解決策として、Seleniumによるスクレイピング技術を用いたり、開発リポジトリをクローンしておくだけでそのリポジトリにおける開発Webページに対して、
            視覚的回帰テストを自動で実行できる機能を実装する必要がある。
      \item 白地以外の背景画像に対応する拡張:\\
            背景が白地でなく、赤色や緑色であったりした場合に、適切に変更箇所に色付きの枠を付けることができない。
            現在の\toolName では、HTMLの変更箇所に赤枠や緑枠をつけて強調表示したり、画像比較で生成した差分画像に赤枠や緑枠をつけて強調表示したりしているが、
            テスト対象とするWebページの背景が白地でなく、赤や緑など強調する色付きの枠と同じであると、適切に強調することができない。
            解決策として、そのようなWebページは視覚的回帰テストの対象としない処理にするか、背景の色や画像内に多く含む色を検出し、その色と区別できる色で変更箇所を強調表示する必要があると考える。
            さらに、他の方法として、色付きの枠をつけて強調表示するのではなく、スライドバーで変更前画像と変更後画像のそのままの表示で切り替えられるようにすることが考えられる。
      \item HTMLコード解析方法の変更による拡張:\\
            複雑なHTML構造をしているWebページに対して、視覚的回帰テストがうまくできない。
            現在の\toolName では、HTML構造が一定の形式に沿っていないと適切にHTMLを解析することができず、視覚的回帰テストに用いることのできない、枠付きHTMLコードを生成してしまう。
            この解決策として、HTMLコードの解析をテキストベースで行うのではなく、DOM解析やより高度な解析技術を用いて、HTMLコードの解析が必要となる。
      \item HTMLコード以外のファイル取得による拡張:\\
            requestsによるHTMLコード取得では、CSSやJavaScriptがリンクで記述されていると、そのリンクを読み込むことができず、適切な枠付きHTMLコードを生成できない。解決策として、
            単一のHTMLコードだけでなく、そのHTMLに用いるCSSやJavascriptのファイルも取得できるようにする必要がある。
      \item レイアウトの不具合箇所の検出アルゴリズムの拡張:\\
            親子関係にある画面要素間に画面要素の重なりが発生しても、その箇所をレイアウトの不具合箇所として強調表示することができない。
            これが起こる原因は、画像比較で差分箇所に付ける枠と、HTML比較で変更箇所に付ける枠が5割以上重なっていない場合に、それらの枠をレイアウトの不具合箇所に付ける枠としているためである。
            解決策として、兄弟関係にある画面要素間に対応するだけでなく、様々なレイアウトの不具合箇所に対して適切な処理を施す必要がある。
\end{itemize}

% \par
% 初期設定を行った状態でテスト対象とするWebページのURLを入力として受け取ることで、
% 以下に示す、\toolName による生成した画像を、Flaskを用いて構築したローカルサーバ上で動作するWebページに、
% 出力し、表示する。
% \begin{itemize}
%     \item Webページの変更前画像と変更後画像
%     \item 画像比較に基づく差分箇所を、色付きの枠で囲むことで強調表示した、Webページの変更前画像と変更後画像
%     \item HTMLコードの変更に基づく影響箇所を、色付きの枠で囲むことで強調表示した、Webページの変更前画像と変更後画像
%     \item レイアウトの不具合箇所を、色付きの枠で囲むことで強調表示した、Webページの変更前画像と変更後画像
% \end{itemize}
% % \par
% 【TODO: ツールの機能の詳細な説明】

% 【TODO: 適用例で確認したこと】

% 【TODO: 実験で分かったこと】

% 【TODO: 今後の課題】



