\chapter{研究の準備}\label{cha:Preparation}
本章では、\toolName を実装する際に必要な前提知識について説明する。

\section{入力対象とするWebページ}\label{sec:target_images}
本研究では、\toolName の入力対象とするWebページは、以下の条件をすべて満たすものとする。
\begin{itemize}
      \setlength{\itemsep}{0pt}
            \setlength{\parsep}{0pt}
      \item 比較対象とするWebページとテスト対象とするWebページの画面サイズが同じである
      \item 比較対象とするWebページとテスト対象とするWebページが著しく異なることがない
      \item 白地の背景である
      \item 赤枠や緑枠を含まない
      \item アニメーションを含まない
      \item 【他にあれば今後追記する】
\end{itemize}

\section{視覚的回帰テスト}\label{sec:vrt}
視覚的回帰テスト (Visual Regression Testing)\cite{VisualRegressionTesting}は、
変更前画像と変更後画像を比較し差分を検出することで、意図しないレイアウトの変更が発生していないことを確認するテスト手法である。\\
% なお、本研究では、比較対象とするWebページの画像を「比較画像」、テスト対象とするWebページの画像を「テスト画像」と定義する。
\par
視覚的回帰テストの基本的な手順は、以下の通りである。
\begin{enumerate}
      \setlength{\itemsep}{0pt}
            \setlength{\parsep}{0pt}
      \item 変更前画像の取得:\\
            比較対象とするWebページの画面をスクリーンショットする。このスクリーンショットした画像を、変更前画像とする。
      \item 変更の適用:\\
            比較対象とするWebページに、画面要素(テキストやボタン、画像など)の追加や削除、既存の画面要素のスタイル変更などを適用する。
      \item 変更後画像の取得:\\
            変更を適用したWebページの画面をスクリーンショットする。このスクリーンショットした画像を、変更後画像とする。
      \item 画像比較:\\
            変更前画像と変更後画像を比較し、差分を検出する。
            % 一部の視覚的回帰テストツール\cite{}は、2つの画像をピクセル単位で比較し、差分を強調表示する。
      \item 結果の評価:\\
            差分検出の結果を確認し、差分が意図したレイアウトの変更であるかを判定する。
            意図しないレイアウトの変更が発生していた場合、修正を行い、再度テストを行う。
\end{enumerate}

\section{レイアウトの不具合}\label{sec:layout effect}
レイアウトの不具合とは、Webページの画面要素が適切にレイアウトされていないことである\cite{LayoutFailureDetection}。
\par
本研究では、以下の3つのレイアウトの不具合を検出する。
\begin{itemize}
      \setlength{\itemsep}{0pt}
            \setlength{\parsep}{0pt}
      \item 画面要素の隠れ:\\
            Webページの画面要素がその画面要素を含むコンテナやビューポートの境界を超えてはみ出している状態。
      \item 画面要素の見切れ:\\
            Webページの画面要素の一部がコンテナやビューポートの境界によって切り取られ、完全には表示されていない状態。
      \item 画面要素の重なり:\\
            Webページの複数の画面要素が重なりあっている状態。
\end{itemize}
% また、以下の画面要素に対して、レイアウトの不具合を検出する。
% \begin{itemize}
%     \setlength{\itemsep}{0pt}
%           \setlength{\parsep}{0pt}
%     \item テキスト
%     \item 画像
%     \item ボタン
%     \item ヘッダー
%     \item フッター
% \end{itemize}
\section{OpenCV}\label{sec:opencv}
OpenCV (Open Source Computer Vision Library)\cite{OpenCV}は、画像や動画に関する処理機能をまとめた、コンピュータビジョン向けのオープンソースのライブラリである。
\par
本研究では、OpenCVに用意されている、imread関数、cvtColor関数、adaptiveThreshold関数、subtract関数、findContours関数、boundingRect関数、rectangle関数、absdiff関数の8つの関数を用いる。

\paragraph{imread関数}
imread関数は、画像ファイルの読み込みを行う関数である。
第一引数に、画像ファイルのパスを指定する。
読み込まれた画像データは、OpenCVで広く使われるデータ形式となる。
これにより、OpenCVに用意されている関数を用いて、画像処理を適用できる。
\par
本研究では、グレースケール化や二値化処理などの画像処理を適用するために、
この関数を使用してWebページの画像や\toolName が生成する画像を読み込む。
\paragraph{cvtColor関数}
cvtColor関数は、画像の色空間を変換する関数である。
第一引数には、色空間の変換を適用する画像を指定し、第二引数には、どの色空間へ変換するかを示すコード\cite{ColorCode}を指定する。
\par
本研究では、画像処理の高速化と精度向上のために、
BGR(青、緑、赤)からGRAY(グレースケール)の色空間への変換を行う。
また、画像処理の適用によって削除箇所や追加箇所を検出した際、色で識別可能にするために、GRAY(グレースケール)からBGR(青、緑、赤)の色空間への変換を行う。
\paragraph{adaptiveThreshold関数}
adaptiveThreshold関数は、画像を複数の小さな区画に切り分けた各々の部分ごとに異なる二値化の閾値を用いて画像の二値化を行う関数である。
第一引数には、二値化を適用するグレースケール画像を指定し、第二引数には、二値化後のピクセルにおける明度の最大値(通常$255$)を指定する。
第三引数には、二値化の閾値を計算する方法を指定し、第四引数には、二値化の方法を指定する。
第五引数には、近傍領域のサイズを指定し、
第六引数には計算された二値化の閾値から引かれる定数を指定する。
\par
本研究におけるadaptiveThreshold関数は、画像の一部が明るく、他の部分が暗い場合においても、全体として均一な二値化画像を生成するために用いる。
\par
また、本研究において、adaptiveThreshold関数の第三引数から第六引数に関しては、以下の固有の値を設定する。
\begin{itemize}
      \setlength{\itemsep}{0pt}
            \setlength{\parsep}{0pt}
      \item 二値化の閾値を計算する方法:\\
            近傍領域内のピクセル平均値(近傍領域内のピクセル合計値 $\div$ 近傍領域内のピクセル数)に基づいて閾値を計算するcv2.ADAPTIVE\_THRESH\_MEAN\_C\cite{AdaptiveThreshold}に設定する。
      \item 二値化の方法:\\
            二値化の閾値以下のピクセル値をピクセルの明度の最大値に変換し、
            それ以外のピクセル値を$0$に変換する二値化方法であるcv2.THRESH\_BINARY\_INV\cite{Threshold}に設定する。
      \item 近傍領域のサイズ:\\
            $11$に設定する。
      \item 計算された二値化の閾値から引かれる定数:\\
            $2$に設定する。
\end{itemize}
% \paragraph{bitwise\_not関数}
% bitwise\_not関数は、画像の各ピクセル値のビットを反転する関数である。
% ビット反転の例として、黒ピクセル($0$のピクセル値)は白ピクセルに(ピクセル値が$255$に)、白ピクセル($255$のピクセル値)は黒ピクセルに(ピクセル値が$0$に)反転する。
% 第一引数にビット反転を行う画像を指定する。
\paragraph{subtract関数}
subtract関数は、2つの画像間の対応するピクセル値を比較し、差分画像を返す関数である。
第一引数には、比較の基準となる画像を指定する。
第二引数には、第一引数の画像と比較する画像を指定する。この画像の各ピクセル値が、第一引数で指定する画像の対応するピクセル値から引かれる。
% (例: 二値化前の画像AとBが元々Webページの画像であり、Webページ内のテキストやボタンなどの画面要素が二値化後の画像で白ピクセル、それ以外の背景などが黒ピクセル領域となっている画像)
例として、二値化画像AとBはそれぞれ白ピクセルに特徴がある画像とし、
subtract関数の第一引数に二値化画像Aを、
第二引数に二値化画像Bをそれぞれ指定した場合のsubtract関数のピクセル値の比較処理を、以下に示す。
\par
\begin{itemize}
      \setlength{\itemsep}{0pt}
            \setlength{\parsep}{0pt}
      \item 共通の白ピクセル:\\
            二値化画像AとBの両方で白ピクセルの場合、subtract関数は$255$(Aの白ピクセル値)から$255$(Bの白ピクセル値)を引く。
            差は$0$(黒ピクセル値)になり、共通の白ピクセルは黒ピクセルとなる。
      \item Aにのみ存在する白ピクセル:\\
            二値化画像Aは白ピクセルで、二値化画像Bは黒ピクセルの場合、subtract関数は$255$(Aの白ピクセル値)から$0$(Bの黒ピクセル値)を引く。
            差は$255$(白ピクセル値)になり、Aにのみ存在する白ピクセルはそのまま残る。
      \item Bにのみ存在する白ピクセル:\\
            二値化画像Aは黒ピクセルで、二値化画像Bは白ピクセルの場合、subtract関数は$0$(Aの黒ピクセル値)から$255$(Bの白ピクセル値)を引く。
            差は$-255$になるが、subtract関数は負の値を$0$(黒ピクセル値)として処理するため、Bにのみ存在する白ピクセルは黒ピクセルとなる。
      \item 共通の黒ピクセル:\\
            二値化画像AとBの両方で黒ピクセルの場合、subtract関数は$0$(Aの黒ピクセル値)から$0$(Bの黒ピクセル値)を引く。
            差は$0$(黒ピクセル値)になり、共通の黒ピクセルはそのまま黒ピクセルとなる。
\end{itemize}
上記の処理の結果、二値化画像Aのみに存在する白ピクセルを抽出した差分画像が生成できる。
\par
本研究では、二値化処理を行った、Webページの変更前画像と変更後画像をそれぞれ交互に1回ずつsubtract関数の第一引数と第二引数に指定し実行することで、
変更前画像から削除された箇所を抽出した差分画像と、変更後画像に追加された箇所を抽出した差分画像の2つの差分画像を生成する。
\paragraph{dilate関数}
dilate関数は、二値化画像内の白ピクセルを拡大する(膨張処理を行う)関数であり、ノイズ除去やオブジェクトの形状とサイズの強調に役立つ。
カーネルと呼ばれる特定の形状とサイズの小さなフィルタを使用し、フィルタを画像上でスライドし各ピクセルに対して、
フィルタ内の最大値をそのピクセルに割り当てていくことで、白ピクセルを拡大する。
第一引数に、膨張処理を行う画像を指定する。
第二引数に、カーネルのサイズと形状を指定する。
第三引数に、膨張処理の適用回数を指定する。
\par
本研究において、dilate関数の第二引数と第三引数に関しては、以下の固有の値を設定する。
\begin{itemize}
      \setlength{\itemsep}{0pt}
            \setlength{\parsep}{0pt}
      \item カーネルのサイズと形状:\\
            5 $\times$ 5ピクセルの正方形に設定する。
      \item 膨張処理の適用回数:\\
            $6$に設定する。
\end{itemize}
\par
本研究におけるdilate関数は、subtract関数で生成した差分画像内の、削除された箇所または追加された箇所の形状とサイズを強調することで、
影響箇所(\ref{cha:Function}章で後述)を枠で囲む粒度と同等の粒度で、後述するrectangle関数が差分箇所(\ref{cha:Function}章で後述)である
削除された箇所と追加された箇所を枠で囲むことを目的としている。
% この目的の結果、差分箇所と影響箇所の比較の際に、レイアウトの副作用箇所(\ref{cha:Function}章で後述)の検出精度を高めることができる。
% 差分箇所(\ref{cha:Function}章で後述)を枠で囲む粒度を、影響箇所(\ref{cha:Function}章で後述)を枠で囲む粒度に近づけることができるため、
% レイアウトの不具合箇所の検出に役立つ。
% 本研究では、差分箇所を枠で囲む粒度を、影響箇所を枠で囲む粒度と合わせるために用いる。
\paragraph{findContours関数}
findContours関数は、画像から輪郭を検出し、検出した各輪郭の輪郭データを格納する輪郭リストを返す。
第一引数に輪郭抽出する画像(通常は二値化画像)を指定する。
第二引数に輪郭構造の取得方法を指定する。
第三引数に、輪郭座標の取得方法を指定する。
\par
本研究において、findContours関数の引数に指定するパラメータを、以下に示す。
\begin{itemize}
      \setlength{\itemsep}{0pt}
            \setlength{\parsep}{0pt}
      \item 輪郭構造の取得方法:\\
            画像内の一番外側の白の輪郭のみを取得するRETR\_EXTERNAL\cite{RetrExternal}を用いる。
      \item 輪郭の形成方法:\\
            端点のみで輪郭を形成するCHAIN\_APPROX\_SIMPLE\cite{ChainApproxSimple}を用いる。
\end{itemize}
\par
% また、本研究におけるfindContours関数は、後述するboundingRect関数で
% Webページの変更前画像から削除された箇所の輪郭座標を、
% Webページの変更後画像に追加された箇所の輪郭座標を取得するために、輪郭データを取得する。
\paragraph{boundingRect関数}
boundingRect関数は、輪郭データにもとに、その輪郭を完全に囲む最小の矩形(バウンディングボックス)を計算する関数である。
第一引数は、findContours関数で取得した輪郭リストの各要素である輪郭データを指定する。
% なお、輪郭データは、単一の輪郭を表す配列であり、この配列に輪郭を形成する点の集合を含む。
\par
本研究におけるboundingRect関数は、矩形の左上の点のx座標とy座標、さらに矩形の幅と高さを取得するために用いる。
これにより、輪郭を取り囲む矩形の位置とサイズを取得できる。
なお、画像の座標系は、画像の左上隅画素を原点 (0,0) とし、右方向をX軸の正の向き、下方向をY軸の正の向きと定義する。
\paragraph{rectangle関数}
rectangle関数は、画像に矩形を描画する関数である。
第一引数に描画先の画像、第二引数に矩形の左上の点の座標、第三引数に矩形の右下の点の座標、
第四引数に矩形の色、第五引数に矩形の線の太さを指定する。
\par
本研究におけるrectangle関数は、画像上に、指定した位置と大きさの矩形を指定した色で描画し、特定の領域を視覚的に強調するために用いる。
\paragraph{absdiff関数}
absdiff関数は、2つの配列または画像間の絶対値の差を計算する関数である。
第一引数に、1つ目の入力配列または画像を指定する。
第二引数に、2つ目の入力配列または画像を指定する。この入力配列または画像は、1つ目の入力配列または画像と同じサイズにし、同じデータ型にする必要がある。
\par
本研究におけるabsdiff関数は、入力された2つの配列または画像の各ピクセル値の絶対差を計算し、その結果を出力配列または画像として返す。

\section{Pillow}\label{sec:pillow}
Pillow\cite{Pillow}は、Python\cite{Python}の画像処理ライブラリの1つで、画像の読み込み、処理、保存などの機能を提供する。
本研究では、Pillowに用意されている、Image.open関数とリサンプリングフィルタであるImage.LANCZOSを用いる。
\paragraph{Image.open関数}
Image.open関数\cite{ImageModule}は、指定したパスの画像ファイルを開き、PillowライブラリによるImageクラスのインスタンスを生成して返す。
このImageクラスのインスタンスに対して、画像のサイズ変更、回転、色調整など、様々な画像処理操作が可能である。
\paragraph{Image.LANCZOS}
Image.LANCZOSは、画像のサイズ変更時に使用されるリサンプリングフィルタの1つである\cite{LANCZOS}。
LANCZOSフィルタは、画像を拡大する際に細部やエッジを維持できるため、高解像度画像の生成処理に適している。
\par
本研究におけるImage.open関数とImage.LANCZOSは、画像比較部の高解像度画像生成処理(\ref{subsec:Generate_high_images}節を参照)の際に、
比較画像とテスト画像から、高解像度比較画像と高解像度テスト画像を生成するために用いる。

\section{NumPy}\label{sec:numpy}
NumPy\cite{NumPy}は、Pythonにおいて数値計算を効率的に行うための拡張モジュールである。
効率的な数値計算を行うための型付きの多次元配列のサポートをPythonに加えるとともに、それらを操作するための大規模な高水準の数学関数ライブラリを提供する。
\par
本研究では、NumPyに用意されている、np.zeros関数を用いる。
\paragraph{np.zeros関数}
np.zeros関数\cite{np.zeros}は、$0$で初期化した画像(黒画像)を生成するために用いる。
第一引数に、生成する配列の形状を指定する。
第二引数に、配列のデータ型を指定する。
なお、配列の形状とは、生成する配列の次元数と各次元数における要素数のことである。

\section{Selenium WebDriver}\label{sec:Selenium_WebDriver}
Selenium WebDriver\cite{SeleniumWebDriver}は、Seleniumプロジェクト\cite{Selenium}の一部であり、Webブラウザを直接制御するためのAPIを提供する。
Selenium WebDriverのAPIを用いてテストスクリプトを実行することで、指定したWebページにアクセスし、要素のクリックやページ遷移などのユーザ操作を自動で行う。

\section{requests}\label{sec:requests}

\section{BeautifulSoup}\label{sec:beautifulsoup}
BeautifulSoup\cite{BeautifulSoup}は、HTMLやXMLファイルからデータを取り出すためのPythonライブラリである。
このライブラリを利用することで、タグの検索や属性の取得、テキストの抽出などを容易に行うことができる。
\par
本研究では、BeautifulSoupに用意されている、find\_all関数とget\_text関数を用いる。
\paragraph{BeautifulSoupの初期化}
BeautifulSoupに用意されている関数を用いるには、まず、BeautifulSoupオブジェクトを初期化する必要がある。
BeautifulSoupオブジェクトとは、HTMLデータを解析し、操作可能な形式に変換したデータ構造である。
このオブジェクトの初期化には、解析するHTMLデータと使用するパーサーを指定する。
\par
本研究では、使用するパーサーに、HTMLデータを解析するパーサーである"html.parser"\cite{BeautifulSoup}を用いる。
\paragraph{find\_all関数}
find\_all関数は、指定したタグに一致するすべての要素を検索し、それらをリストとして返す。
\paragraph{get\_text関数}
get\_text関数は、指定したタグのテキスト内容を返す。

\section{difflib}\label{sec:difflib}
difflib\cite{difflib}は、Pythonでテキストデータ間の差分を検出し、処理するための標準モジュールである。
このモジュールは、異なるテキストデータ間の差分を識別し、それらの違いを詳細に比較する機能を提供する。
\par
本研究では、difflibに用意されている、Differクラスとcompare関数を用いる。
\paragraph{Differクラス}
Differクラスは、2つのテキスト間の差分を行単位で計算し、詳細な差分情報を生成するためのクラスである。
このクラスは、テキストの各行を個別に比較し、追加、削除、変更された行を特定する。
\paragraph{compare関数}
compare関数は、Differクラスのインスタンスを使用して、2つの文字列リスト間の差分を計算する。
この関数は、2つの文字列リストを引数として受け取り、差分結果を行単位で返す。
差分結果には、追加された行が'+ '記号で、削除された行が'- ' 記号で、変更のない行がスペースでマークされる。
また、'? ' 記号を用いて、変更があった行についての詳細な差異を示す。
\par
本研究では、'? '記号がある行については、解析の対象とはせず、その他の行を解析対象とする。
% \begin{itemize}
%       \setlength{\itemsep}{0pt}
%             \setlength{\parsep}{0pt}
%       \item "- ":\\
%             行はシーケンス1にのみ存在する。
%       \item "+ ":\\
%             行はシーケンス2にのみ存在する。
%       \item "? ":\\
%             行は入力シーケンスのどちらにも存在しない(どの文字がどのように変更したかという差分の詳細)。
% \end{itemize}
% 差分コードの例を、ソースコード\ref{lst:diff_html_?}に示す。
% \begin{lstlisting}[language=HTML, caption=差分コードの例, label=lst:diff_html_?]
%   <header>
% -     <h1>Single Content Web Page</h1>
% +         <h1>Single Content Web Page with layout bug</h1>
% ?                                    ++++++++++++++++

%   </header>
% \end{lstlisting}

\section{Flask}\label{sec:Flask}
Flask\cite{Flask}は、Pythonで構築されたWebフレームワークであり、WebサーバやWebアプリケーションを構築できる。
このフレームワークは、Werkzeug\cite{Werkzeug}とJinja2\cite{Jinja}に基づいている。
Werkzeugは、WebサーバとWebアプリケーション間のリクエストやレスポンスを処理するWSGI(Web Server Gateway Interface)ツールキットである。
なお、WSGI\cite{WSGI}とは、WebサーバとWebアプリケーション間の接続規格である。
Jinja2は、Pythonで使用されるテンプレートエンジン\cite{TemplateEngine}で、Webページの動的なレンダリングを可能にする。
\par
本研究では、Flaskを用いてローカルサーバを構築し、以下を実現する。
\begin{itemize}
      \setlength{\itemsep}{0pt}
            \setlength{\parsep}{0pt}
      \item \toolName による視覚的回帰テストで生成した画像(\ref{subsec:MixVRT_IO}節で後述)の確認が効率的に行えるWebページを開発者に提供する。
      \item HTML比較部(\ref{sec:Affected_area_extraction}節で後述)で生成した枠付きHTMLコードをレンダリングし、
            Flaskを用いて定義した特定のURLへのアクセスを通じて、レンダリングを行った枠付きWebページを表示する。
\end{itemize}
なお、枠付きWebページを表示する目的は、Selenium WebDriverを用いて、枠付きWebページの画像を取得するためである。

\section{\toolName の環境構築}\label{sec:MixVRT_env_gen}
\toolName の環境構築には、Dockerコンテナ\cite{DockerContainer}を使用する。
具体的には、ローカルサーバ用、Python処理用、およびSeleniumのGoogle Chrome用の、3つのDockerコンテナ\cite{DockerContainer}を使用する。
Docker Compose\cite{DockerCompose}を用いて、開発者はこれらのコンテナを同時に起動する。
\par
ローカルサーバ用コンテナは、flaskを用いたローカルサーバを提供する。
Python処理用コンテナは、Pythonを用いた画像処理やHTMLコード解析処理を行う。
SeleniumのGoogle Chrome用コンテナは、Webブラウザの自動操作を実行するためのGoogle Chrome環境を提供する。
なお、これらのコンテナに必要な設定と依存関係は、Dockerfileとdocker-compose.ymlに定義する。
Dockerfileは、Dockerコンテナの構築に必要な指示を記述したファイルである。
docker-compose.ymlは、複数のコンテナを管理するための設定ファイルである。

\section{\texorpdfstring{$<$p$>$タグ内のテキストの整形}{pタグ内のテキストの整形}}\label{sec:text_change}
取得したHTMLコード内の$<$p$>$タグ内のテキストを整形する。
整形する目的は、変更前後のHTMLコードを行単位で比較した際に、
$<$p$>$タグ内のテキストのように複数行にわたって書かれる可能性がある画面要素を、
変更箇所として検出するためである。
body要素内の変更箇所検出は、変更前後のHTMLコードを行単位で比較し、変更があるかどうかを判定する。
しかし、$<$p$>$タグ内の複数行にわたるテキストがあり、そのテキスト中の一部が変更されていた場合に、
その行中に開始タグが無いと枠を囲むCSSクラスを追加することができない。
そのため、複数行にわたってテキストが書かれる可能性の高い$<$p$>$タグのみを対象に、
1行で収まるように$<$p$>$タグ内のテキストを整形する。
具体例として、$<$p$>$タグ内のテキストの整形前と整形後の例を、以下のソースコード\ref{lst:text_bf}とソースコード\ref{lst:text_af}に示す。
\begin{figure}[tp]
      \begin{lstlisting}[language=HTML, caption=$<$p$>$タグ内のテキストの整形前の例, label=lst:text_bf]
<p>A week consists of seven days. 
    It begins with Monday, marking the start of a new week. 
    Tuesday follows, leading to Wednesday, which is the middle of the week. </p>
\end{lstlisting}
\end{figure}

\begin{figure}[tp]
      \begin{lstlisting}[language=HTML, caption=$<$p$>$タグ内のテキストの整形後の例, label=lst:text_af]
<p>A week consists of seven days. It begins with Monday, marking the start of a new week. Tuesday follows, leading to Wednesday, which is the middle of the week. </p>
\end{lstlisting}
\end{figure}