\chapter{研究の準備}\label{cha:Preparation}

\section{視覚的回帰テスト}\label{sec:vrt}
視覚的回帰テスト【DO:全角かっこ⇒半角かっこ】【TODO:参考文献のリンク張る】(visual regression testing)は、Webページの【TODO:視覚的要素⇒見た目】見た目が予期せず変更されていないことを確認するテスト手法である。
視覚的回帰テストの基本的な手順は以下の通りである。
\begin{enumerate}
    \setlength{\itemsep}{0pt}
          \setlength{\parsep}{0pt}
    \item 【TODO:ベースライン画像の説明 or 別の表現】ベースライン画像と比較対象画像の作成
    \item 画像比較による差分検出
    \item 結果の評価
\end{enumerate}
\section{レイアウトの不具合}\label{sec:layout effect}

\section{OpenCV}\label{sec:OpenCV}

\section{Selenium WebDriver}\label{sec:Selenium_WebDriver}
Selenium WebDriver【TODO:参考文献のリンク張る】は、Seleniumプロジェクトの一部であり、Webブラウザを直接制御するためのAPIを提供する。
Selenium WebDriver【DO:WebDriver⇒Selenium WebDriver】のAPIを用いてテストスクリプトを実行することで、指定したWebページにアクセスし、要素のクリックやページ遷移などのユーザ操作を自動で行う。
\section{requestsモジュール}\label{sec:requests}

\section{difflibモジュール}\label{sec:difflib}

\section{Flask}\label{sec:Flask}