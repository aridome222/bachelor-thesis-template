\chapter{研究の準備}\label{cha:Preparation}

\section{視覚的回帰テスト}\label{sec:vrt}
視覚的回帰テスト(visual regression testing)は、Webページの視覚的要素が予期せず変更されていないことを確認するテスト手法である。
視覚的回帰テストの基本的な手順は以下の通りである。
\begin{enumerate}
    \setlength{\itemsep}{0pt}
          \setlength{\parsep}{0pt}
    \item ベースライン画像と比較対象画像の作成
    \item 画像比較による差分検出
    \item 結果の評価
\end{enumerate}
\section{レイアウトの不具合}\label{sec:layout effect}

\section{OpenCV}\label{sec:OpenCV}

\section{Selenium WebDriver}\label{sec:Selenium_WebDriver}
Selenium WebDriver は、Seleniumプロジェクトの一部であり、Webブラウザを直接制御するためのAPIを提供する。
WebDriverのAPIを用いてテストスクリプトを実行することで、指定したWebページにアクセスし、要素のクリックやページ遷移などのユーザ操作を自動で行う。
\section{requestsモジュール}\label{sec:requests}

\section{difflibモジュール}\label{sec:difflib}

\section{Flask}\label{sec:Flask}