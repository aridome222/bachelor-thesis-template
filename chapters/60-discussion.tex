\chapter{考察}\label{cha:Discussion}
本研究では、レイアウトの副作用箇所を発見するのにかかる時間の削減を目的とした、差分強調ツール\toolName を試作した。
本章では、まず、試作した\toolName と人手で、それぞれレイアウトの副作用を見つけるのにかかった時間と発見率を比較し、
\toolName の有用性を評価する。
次に、\toolName と関連研究を比較する。
最後に、\toolName の問題点とその解決策について述べる。


\section{\toolName の有用性に関する評価}\label{sec:evalue_usefulness}
本研究で試作した\toolName の有用性を評価するために、以下に示す2つの評価を行った。
\begin{itemize}
    \item レイアウトの副作用箇所の発見時間に関する評価
    \item レイアウトの副作用箇所の発見率に関する評価
\end{itemize}
以降、2つの評価結果について、それぞれ説明する。


\subsection{レイアウトの副作用発見時間に関する評価}\label{subsec:evalue_required_time}
レイアウトの副作用箇所の発見時間に関する評価を行うために、
\toolName を使用した場合と人手の場合で、レイアウトの副作用箇所の発見にかかる作業時間を比較した。
\toolName を使用した場合の計測時間は、\toolName の実行時間 + \toolName による実行結果確認画面を見た瞬間からレイアウトの副作用箇所のを見つけるまでにかかった時間を測った。
人手の場合は、実験に用いたWebページの画像を見てからレイアウトの副作用箇所を見つけるまでにかかった時間を測った。

\subsection{レイアウトの副作用箇所の発見率に関する評価}\label{subsec:evalue_accuracy}
レイアウトの副作用箇所の発見率に関する評価を行うために、
\toolName を使用した場合と、人手の場合で、レイアウトの副作用箇所の発見率を比較した。



\section{関連研究}\label{sec:relation_research}



\section{\toolName の問題点とその解決策}\label{sec:AWSEL_problems}