\chapter{考察}\label{cha:Discussion}
本研究では、レイアウトの不具合箇所の可視化を目的とした、視覚的回帰テストツール\toolName を試作した。
本章では、まず、試作した\toolName と人手でレイアウトの不具合を見つけるのにかかった時間と検出率を比較し、
\toolName の有用性を評価する。
次に、\toolName と関連研究を比較する。
最後に、\toolName の問題点とその解決策について述べる。


\section{\toolName の有用性に関する評価}\label{sec:evalue_usefulness}
本研究で試作した\toolName の有用性を評価するために、以下に示す2つの評価を行った。
\begin{itemize}
    \item レイアウトの不具合発見時間に関する評価
    \item レイアウトの不具合発見率に関する評価
\end{itemize}
以降、2つの評価結果について、それぞれ説明する。


\subsection{レイアウトの不具合発見時間に関する評価}\label{subsec:evalue_required_time}
レイアウトの不具合発見時間に関する評価を行うために、
\toolName を使用した場合と、人手の場合で、レイアウトの不具合の発見にかかる作業時間の比較を行った。
\toolName を使用した場合の計測時間は、\toolName の実行時間 + \toolName による実行結果確認画面を見た瞬間からレイアウトの不具合を見つけるまでにかかった時間 を測った。
人手の場合は、実験に用いたWebページの画像を見た瞬間からレイアウトの不具合を見つけるまでにかかった時間を測った。

\subsection{レイアウトの不具合発見率に関する評価}\label{subsec:evalue_accuracy}
レイアウトの不具合発見率に関する評価を行うために、
\toolName を使用した場合と、人手の場合で、レイアウトの不具合の発見率の比較を行った。



\section{関連研究}\label{sec:relation_research}



\section{\toolName の問題点とその解決策}\label{sec:AWSEL_problems}