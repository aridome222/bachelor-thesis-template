\chapter{はじめに}\label{cha:Introduction}
インターネットの利用率は年々増加している\cite{Soumusyou}。それに伴い、WebサイトやWebアプリケーション開発の需要も高まっている。
近年のWebアプリケーション開発では、JavaScriptなどのプログラミング言語を用いて、フロントエンドの開発を行うことが一般的であり【TODO:参考文献】、
そのソースコードをHTMLコードに変換することで、Webブラウザ上にページを描画する。
この開発の際、開発者の意図しない箇所でレイアウトの変更が生じる問題がある。
\par
この問題を解決するために、
視覚的回帰テスト(Visual Regression Test)\cite{VisualRegressionTesting}という手法があり、
開発者がWebページの要素を追加・削除・変更した際に、
意図しない箇所でレイアウトの変更が発生していないことを確認できる。
視覚的回帰テストを行うことで
変更前後のWebページを画像比較したレイアウトの差分を確認し、
変更後のWebページのみを確認するだけでは発見することが困難なレイアウトにおける不具合を、
検出することが可能となる。
既存の視覚的回帰テストを対象とした研究では、変更前後の画像のみを比較し、画面上の差分をすべて表示するものが多い。
しかし、
この差分には、開発者の意図した変更も含まれる。
意図しない変更を確認する際には、
表示されたすべての差分を注意深く確認し、
その差分が意図しない変更であるかどうかを、
HTMLコードを見て判断する必要がある。
% しかし、描画されるページのHTMLコードには変更がないにも関わらず、画面上では画面要素が変更されてしまう問題がある。
% この問題に対処するためには、開発者は、Webアプリケーションの変更前後で、HTMLコードの差分とレイアウトの差分における整合性を確認する必要がある。
この意図しない変更の確認には、手間と時間がかかる。
\par
そこで本研究では、
レイアウトの差分から意図しない変更の確認にかかる時間の削減を目的として、
レイアウトの不具合箇所の可視化機能を持つ視覚的回帰テストツール\toolName(Mix Visual Regression Testing)を試作する。
\par
\toolName は、以下の3つを可視化する機能を持つ。
\begin{itemize}
      \item 画像比較に基づく差分箇所:\\
            Webページの変更前画像とWebページの変更後画像を比較して、
            変更前のWebページから削除された画面要素と変更後のWebページに追加された画面要素を可視化する
      \item HTMLコードの変更に基づく変更箇所:\\
            変更前後のWebページのHTMLコードを比較して、
            HTMLコードにおけるbody要素内の変更とstyle要素内の変更のどちらか、
            または両方の変更が適用された画面要素を可視化する
      \item レイアウトの不具合箇所:\\
            Webページの変更前画像と変更後画像で、変更箇所による影響によって、
            HTMLコードを変更していない画面要素にレイアウトの変更が生じた画面要素を可視化する
\end{itemize}
% 本論文では、この問題を不整合が発生する場合がある。

% ソフトウェアの大規模化や複雑化に伴い、ソフトウェアの信頼性を確保するために、
% 最初のリリース時だけでなく、
% バージョンアップするたびに、テストを行う必要がある\cite{RegionDetect}。
% この際、バージョンアップ前に動いていた機能や性能が損なわれていないかを確認するテスト手法として、回帰テストが存在する。
% 特に、WebアプリケーションのようなGUIアプリケーションの回帰テストでは、アプリケーション画面が正しく表示されていることを、
% 認する回帰テストである視覚的回帰テスト(Visual Regression Test)がある。
% バージョンアップによるレイアウトの変更に伴う、レイアウトの崩れ
% \par
% Webアプリ開発では、HTML要素を変更していないのに、画面上でそのHTML要素が変更されてしまうというレイアウトの不具合がある。
% 上記のレイアウトの不具合を見つけるには、HTMLコードの差分とレイアウトの差分との間における整合性の確認が必要である。






% Webページの変更前後における画像比較に基づく差分箇所は、開発者の意図した変更であるかどうかを確認することができない。
% 差分箇所が意図した変更であるかどうかを開発者が確認するためには、
% 画像比較に基づく差分箇所から得られる視覚情報と、HTMLコードを見て得られるコード情報を
% 組み合わせることが効果的である。
% このデバッグ作業により、差分箇所が意図した変更であるかどうかを判定できるが、
% コード情報と視覚情報間の整合性の確認には、時間がかかる。
% \par
% そこで本研究では、コード情報と視覚情報間における整合性の確認にかかる時間の削減を目的として、
% レイアウトの副作用箇所強調表示(可視化)機能を持つ視覚的回帰テストツール\toolName を試作する。
% \par
% 【下記は2章に書く】\\
% 画像比較に基づく差分箇所には、意図したレイアウトの変更と意図しないレイアウトの変更がある。
% 意図したレイアウトの変更はHTMLコードに基づいており、意図しないレイアウトの変更はHTMLコードに基づいていない。
% なお、本研究では、意図しないレイアウトの変更箇所を、「レイアウトの不具合箇所」と定義する。

% \par
% 最初に、目視で変更前後のWebページを見比べて、不具合がないかどうかを見つけることの難しさを記述する。
% \begin{itemize}
%     \item Item 01
%     \item Item 02
% \end{itemize}
% $\\\\\\\\\\$
% 画像ベースの視覚的回帰テストの説明と、その問題点について記述する。
% \begin{itemize}
%     \item Item 01
%     \item Item 02
% \end{itemize}

% 本研究で試作するツールで課題を解決できることについて記述する。
% \begin{itemize}
%     \item Item 01
%     \item Item 02
% \end{itemize}

\par
本論文の構成を、以下に示す。\par
第2章では、本研究に必要となる前提知識を説明する。\par
第3章では、MixVRTの機能と外観について説明する。\par
第4章では、MixVRTの実装について説明する。\par
第5章では、適用例を用いて、MixVRTが正しく動作することを示す。\par
第6章では、MixVRTについて考察する。\par
第7章では、本研究のまとめと今後の課題について述べる。
